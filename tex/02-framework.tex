\section{Relational Framework and Projective Description}
  \label{sec:framework}

  In this section, we summarize the minimal relational and projective framework
  required for the present analysis.
  The detailed construction of this framework is developed in companion works,
  including the relational and spectral reconstruction of effective
  geometry~\cite{Beau2026a}, the formal role of non-injective projection in
  effective descriptions~\cite{Beau2026b}, and the emergence of bounded-response
  regimes relevant for mass, charge, and inertia~\cite{Beau2026c}.
  Here we restrict attention to the structural assumptions that constrain how
  mass, charge, and inertia can consistently arise as effective descriptors.

  \subsection{Underlying relational description}
    \label{subsec:relational-description}

    We assume that effective spacetime observables arise from an underlying
    relational description, in which configurations encode admissible correlations
    between abstract degrees of freedom.
    No spacetime manifold, metric, or field content is assumed at this level.

    Explicit realizations of such relational descriptions, including spectral
    reconstructions of effective geometry from correlation structure alone, have
    been developed in companion work~\cite{Beau2026a}.
    The present paper does not rely on the details of those constructions, but
    adopts their structural conclusion: spacetime geometry is an effective
    description, not a fundamental input.

    The underlying relational description should not be interpreted as a
    hidden-variable completion of existing theories.
    It defines a space of admissible configurations whose effective description is
    intrinsically coarse-grained.

    Configurations at this level are not endowed with local coordinates, temporal
    ordering, or intrinsic geometric meaning.
    These notions arise only when relational configurations admit a stable and
    approximately injective representation.

  \subsection{Projection to effective observables}
    \label{subsec:projection}

    Observable physical quantities are defined through a projection
    \begin{equation}
      \Pi : \Omega \rightarrow \mathcal{O},
    \end{equation}
    where $\Omega$ denotes the space of underlying relational configurations and
    $\mathcal{O}$ the space of effective observables accessible within spacetime
    descriptions.

    A key structural feature of this projection is that it is generally
    non-injective.
    Distinct underlying configurations may correspond to the same effective
    observable state.
    This identification is not an approximation, but an intrinsic property of the
    descriptive mapping.

    The physical consequences of non-injective projection have been analyzed in
    detail in the context of quantum correlations~\cite{Beau2026b}.
    There it was shown that non-injectivity provides a sufficient mechanism for the
    failure of classical probabilistic factorization, without invoking nonlocal
    dynamics or hidden-variable assumptions.

    In the present work, the same projective structure is shown to govern inertial
    and interaction-related quantities.
    Mass, charge, and inertia are therefore treated as effective descriptors
    defined on equivalence classes of underlying relational configurations.

  \subsection{Saturation and bounded response}
    \label{subsec:saturation}

    A central structural requirement of the present framework is that the projection
    $\Pi$ admits an intrinsic saturation.
    Beyond a certain threshold, additional variations in the underlying relational
    configuration cannot be resolved within the effective description.

    This bounded-response behavior has been studied in detail in companion
    work~\cite{Beau2026c}, where it was shown that saturation naturally leads to
    effective Born--Infeld-type dynamics and to the dynamical selection of stable
    geometric regimes.

    Here, saturation is not introduced as a modification of fundamental dynamics.
    It is interpreted as a limitation of the effective description itself,
    reflecting a finite projective resolution.
    As a result, effective responses cannot be extrapolated linearly under
    arbitrarily large relational gradients.

    Operationally, this implies that departures from linear behavior are not
    contingent features, but necessary consequences of finite projective capacity.

  \subsection{Effective descriptors and regime dependence}
    \label{subsec:effective-descriptors}

    Physical quantities such as mass, electric charge, and inertia are treated as
    effective descriptors defined within the projected space $\mathcal{O}$.
    They characterize how the effective description responds to variations of the
    underlying relational configuration.

    These quantities are inherently regime dependent.
    In domains where the projection is approximately injective and unsaturated,
    classical descriptions are recovered and effective responses behave linearly.
    In saturated regimes, departures from linearity arise as a direct consequence
    of projective limitations.

    Importantly, the present framework does not modify the operational content of
    the Standard Model in experimentally tested regimes.
    It provides a structural reinterpretation of existing quantities while
    constraining the conditions under which their linear extrapolation remains
    valid.

    Within this perspective, distinctions between mass, charge, and inertial
    response correspond to different symmetry classes of saturated effective
    behavior.
    This classification is developed in the following sections.
