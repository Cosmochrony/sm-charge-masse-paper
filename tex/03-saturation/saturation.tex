\section{Symmetry Classes of Saturated Effective Responses}
  \label{sec:saturation}

  In this section, we analyze how different effective physical properties emerge as distinct symmetry
  classes of saturated response within the projective framework introduced above.
  The goal is to identify minimal structural criteria that distinguish mass, electric charge, and
  inertial behavior without introducing new fundamental degrees of freedom.

  \subsection{Linear and saturated response regimes}
    \label{subsec:linear-saturated}

    When the projection from underlying relational configurations to effective observables is
    approximately injective, variations at the underlying level are faithfully reflected in the
    effective description.
    In this regime, effective responses scale linearly with the magnitude of relational gradients.

    Such linear behavior underlies the standard formulation of classical field theories.
    Small perturbations lead to proportionate responses, and superposition principles apply.

    However, when relational gradients exceed the resolving capacity of the effective description,
    the projection enters a saturated regime.
    Beyond this threshold, additional variations at the underlying level no longer produce distinct
    effective outcomes.

    This transition from linear to saturated response is structural rather than dynamical.
    It reflects a limitation of descriptive resolution, not a modification of underlying laws.
    Bounded-response regimes of this type are known to give rise to Born--Infeld-like effective dynamics
    in a variety of contexts~\cite{BornInfeld1934,Beau2026c}.

  \subsection{Isotropic saturation and effective mass}
    \label{subsec:isotropic}

    We first consider saturated responses that preserve isotropy in the effective description.
    In this case, saturation suppresses relational variations uniformly in all directions.

    An isotropic inhibition of effective response leads to behavior characteristic of inertial and
    gravitational mass.
    The effective description resists changes in motion independently of direction, yielding a scalar
    parameter that quantifies the degree of saturation.

    From this perspective, mass is not introduced as an intrinsic substance.
    It appears as a measure of how strongly the effective description inhibits relational reconfiguration
    once saturation is reached.

    This interpretation is consistent with the equivalence between inertial and gravitational mass.
    Both correspond to isotropic limitations of effective response, differing only in the context in
    which the response is probed.
    The universality of free fall follows naturally from the symmetry of the saturation mechanism.

  \subsection{Oriented saturation and effective charge}
    \label{subsec:oriented}

    We now turn to saturated responses that break isotropy while preserving locality and stability.
    In this class, saturation is directionally biased or oriented with respect to relational gradients.

    An oriented saturation leads to effective behavior characteristic of electric charge.
    The effective response distinguishes between opposing directions, giving rise to attractive and
    repulsive interactions depending on orientation.

    Charge thus appears as a signed quantity associated with asymmetric saturation of relational flux.
    The existence of both positive and negative charges reflects the presence of two stable orientations
    of the saturated response.

    Importantly, this interpretation does not require introducing charge as a primitive coupling.
    It arises as a symmetry-breaking mode of the same bounded-response mechanism that gives rise to mass.

    The long-range character of electromagnetic interactions follows from the fact that oriented
    saturation affects relational configurations without isotropic suppression, allowing extended
    field-like behavior within the effective description.

  \subsection{Inertia as finite-resolution response}
    \label{subsec:inertia}

    Inertial behavior emerges when changes in motion probe the finite resolution of the effective
    description.
    Acceleration corresponds to a demand for rapid reconfiguration of relational correlations.

    When such reconfiguration exceeds the resolving capacity of the projection, the effective
    description responds with resistance to change.
    This resistance manifests as inertia.

    In this view, inertia is not a fundamental property attached to matter.
    It reflects the finite resolvability of variations in relational configurations within the effective
    description.

    This interpretation aligns with the structural origin of inertial effects discussed in relational
    approaches to dynamics and complements traditional formulations without altering their empirical
    content.

  \subsection{Unified classification}
    \label{subsec:classification}

    The analysis above leads to a unified classification of effective physical properties.

    Isotropic saturated responses correspond to mass.
    Oriented saturated responses correspond to charge.
    Inertial response reflects the finite resolution of effective reconfiguration under acceleration.

    All three arise from the same structural mechanism: saturation of the effective description under
    non-injective projection.
    Their distinction follows from symmetry properties rather than from distinct underlying substances
    or fields.

    This classification preserves the operational definitions of mass and charge used in the Standard
    Model.
    It reinterprets their origin without modifying their dynamical role or empirical predictions.

    In the next section, we examine the consistency of this framework with established phenomenology and
    discuss potential implications and limitations.
