\section{Symmetry Classes of Saturated Effective Responses}
  \label{sec:saturation}

  In this section, we identify the symmetry classes of effective physical behavior
  that necessarily arise once saturation of the projective description is taken
  into account.
  The goal is not to introduce new physical entities, but to show that mass,
  electric charge, and inertial response correspond to the only stable symmetry
  realizations of a bounded effective response, without invoking additional
  fundamental degrees of freedom.

  \subsection{Linear and saturated response regimes}
    \label{subsec:linear-saturated}

    When the projection from underlying relational configurations to effective
    observables is approximately injective, variations at the underlying level are
    faithfully reflected in the effective description.
    In this regime, effective responses scale linearly with the magnitude of
    relational gradients.

    Such linear behavior underlies the standard formulation of classical field
    theories.
    Small perturbations lead to proportionate responses, and superposition
    principles apply.

    However, when relational gradients exceed the resolving capacity of the
    effective description, the projection necessarily enters a saturated regime.
    Beyond this threshold, additional variations at the underlying level no longer
    produce distinct effective outcomes.

    This transition from linear to saturated response is structural rather than
    dynamical.
    It reflects a limitation of descriptive resolution, not a modification of
    underlying laws.
    Bounded-response regimes of this type are known to give rise to Born--Infeld-like
    effective dynamics in a variety of contexts~\cite{BornInfeld1934,Beau2026c}.

  \subsection{Isotropic saturation and effective mass}
    \label{subsec:isotropic}

    We first consider saturated responses that preserve isotropy in the effective
    description.
    In this case, saturation suppresses relational variations uniformly in all
    directions.

    An isotropic inhibition of effective response leads necessarily to behavior
    characteristic of inertial and gravitational mass.
    The effective description resists changes in motion independently of direction,
    yielding a scalar parameter that quantifies the degree of saturation.

    From this perspective, mass is not introduced as an intrinsic substance.
    It appears as a measure of how strongly the effective description inhibits
    relational reconfiguration once saturation is reached.

    This interpretation is consistent with the equivalence between inertial and
    gravitational mass.
    Both correspond to isotropic limitations of effective response, differing only
    in the context in which the response is probed.
    The universality of free fall follows as a direct consequence of the symmetry of
    the saturation mechanism.

  \subsection{Oriented saturation and effective charge}
    \label{subsec:oriented}

    We now consider saturated responses that break isotropy while preserving
    locality and stability of the effective description.
    In this class, saturation is directionally biased with respect to relational
    gradients.

    Such oriented saturation necessarily leads to behavior characteristic of
    electric charge.
    The effective response distinguishes between opposing directions, giving rise
    to attractive and repulsive interactions depending on orientation.

    Charge thus appears as a signed quantity associated with asymmetric saturation
    of relational flux.
    The existence of both positive and negative charges reflects the presence of two
    stable orientations of the saturated response.

    Importantly, this interpretation does not require introducing charge as a
    primitive coupling.
    It arises as a symmetry-breaking mode of the same bounded-response mechanism
    that gives rise to mass.

    The long-range character of electromagnetic interactions follows from the fact
    that oriented saturation does not induce isotropic suppression, allowing
    extended field-like behavior within the effective description.

  \subsection{Inertia as finite-resolution response}
    \label{subsec:inertia}

    Inertial behavior emerges when changes in motion probe the finite update
    capacity of the effective description.
    Acceleration corresponds to a demand for rapid reconfiguration of relational
    correlations within the projected space.

    When such reconfiguration remains well below the projective capacity, the
    effective response is linear.
    In this regime, resistance to acceleration scales proportionally with the
    applied stress, reproducing the standard inertial relation as a linear-response
    approximation.

    As the required rate of reconfiguration approaches the resolving limit of the
    projection, the effective response saturates.
    Beyond this point, additional relational variation cannot be resolved
    instantaneously, and the projected dynamics exhibits a bounded update behavior.

    In this view, inertia is not a fundamental property attached to matter.
    It is the linear-response limit of a bounded projective update process,
    reflecting the finite resolvability of variations in relational configurations
    within the effective description.

    This interpretation preserves all empirical content of relativistic kinematics
    while reinterpreting inertial resistance as a consequence of finite descriptive
    capacity rather than a primitive dynamical postulate.

  \subsection{Unified classification}
    \label{subsec:classification}

    The analysis above yields a classification of the \emph{minimal} effective physical responses
    permitted by a bounded projective description under a single scalar saturation constraint.

    Within this scope, three stable response classes are singled out.
    Isotropic saturated responses correspond to mass-like behavior.
    Oriented saturated responses correspond to charge-like behavior.
    Finite-resolution resistance to reconfiguration corresponds to inertia.

    All three arise from the same structural mechanism: saturation of the effective description
    under non-injective projection.
    Their distinction follows from symmetry properties of the saturation manifold rather than
    from distinct underlying substances or independent fundamental fields.

    This classification establishes existence rather than completeness.
    It shows that, once bounded projective capacity is assumed, mass-like, charge-like, and
    inertial responses necessarily appear as the minimal symmetry classes compatible with a
    scalar saturation bound.
    It does not yet claim to exhaust all possible gauge structures.

    In particular, partially anisotropic or higher-rank saturation patterns are not excluded.
    Such patterns correspond to saturation manifolds with higher-dimensional internal
    degeneracies.
    For an $N$-component relational multiplet subject to a scalar bound
    $\rho^2=\vec{\phi}\cdot\vec{\phi}$, the symmetry group preserving saturation is $O(N)$, whose
    connected component $SO(N)$ becomes non-abelian for $N\geq 3$.
    Local compatibility of such degeneracies naturally leads to non-abelian gauge redundancies.

    Accordingly, the present work identifies the minimal rank-one realization of bounded-response
    symmetry classes and delineates a concrete extension path toward higher-rank oriented
    saturation.
    The embedding of the full Standard Model gauge group is deferred to a dedicated analysis of
    multi-channel projection and stability conditions.

    In the following section, we examine the consistency of this framework with established
    phenomenology and show that bounded-response symmetry classes imply necessary limits on
    linear extrapolations.
