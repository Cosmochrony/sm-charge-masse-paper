\section{Discussion, Limitations, and Outlook}
  \label{sec:discussion}

  In this final section, we summarize the conceptual implications of the proposed framework, clarify
  its limitations, and outline possible directions for future investigation.
  The emphasis remains on interpretation rather than on the introduction of new dynamical content.

  \subsection{Conceptual implications}
    \label{subsec:implications}

    The central implication of the present analysis is that mass, electric charge, and inertia need not
    be treated as fundamentally distinct physical primitives.
    Instead, they may be understood as different symmetry realizations of a single structural mechanism,
    namely the saturation of effective response under non-injective projection.

    From this perspective, the apparent diversity of physical properties reflects differences in how the
    effective description responds to relational variation, rather than differences in underlying
    substance.
    Scalar, oriented, and finite-resolution responses correspond respectively to mass, charge, and
    inertial behavior.

    This interpretation provides a unified conceptual basis for the equivalence between inertial and
    gravitational mass and for the signed nature of electric charge, without modifying their operational
    definitions.
    It also clarifies why these quantities enter physical laws as parameters governing response rather
    than as directly observable entities.

  \subsection{Relation to existing foundational approaches}
    \label{subsec:relation}

    The framework developed here is compatible with a wide range of foundational approaches in which
    spacetime and physical observables are regarded as effective constructs.
    It does not rely on a specific microscopic ontology and can be embedded in different relational,
    background-independent, or pre-geometric settings.

    Unlike approaches that postulate new degrees of freedom or modified dynamics, the present work
    operates at the level of description.
    Its contribution is to isolate a minimal structural mechanism capable of accounting for several
    distinct physical notions within a single conceptual scheme.

    In this sense, the framework complements rather than replaces existing theories.
    It provides an interpretative layer that may coexist with standard formulations of quantum field
    theory and relativity.

  \subsection{Limitations of the present work}
    \label{subsec:limitations}

    The present analysis is subject to several important limitations.

    First, no microscopic dynamics of the underlying relational description is specified.
    As a result, the framework does not predict numerical values for masses, charges, or coupling
    constants.
    These remain empirical inputs at the level of effective theories.

    Second, the analysis is qualitative in nature.
    While symmetry arguments and structural considerations motivate the proposed classification, no
    explicit derivation of Standard Model parameters is attempted.

    Third, the framework does not address questions related to particle generations, flavor structure,
    or symmetry breaking patterns within the Standard Model.
    Its scope is restricted to the conceptual status of mass, charge, and inertia as effective
    descriptors.

    Finally, the saturation mechanism invoked here is not directly observable in particle physics
    experiments, as all accessible regimes remain well within the linear-response domain.
    The framework therefore does not offer immediate experimental tests at high-energy scales.

  \subsection{Possible extensions and outlook}
    \label{subsec:outlook}

    Despite these limitations, the framework suggests several directions for further investigation.

    One possible extension is a more detailed analysis of how different symmetry classes of saturation
    could be embedded in concrete relational or spectral models, potentially linking qualitative
    classification to quantitative constraints.

    Another direction concerns the relation between saturation mechanisms and known bounded-response
    effects in quantum field theory, such as strong-field phenomena.
    Clarifying these connections may help identify regimes in which saturation becomes operationally
    relevant.

    At a more conceptual level, the framework invites reconsideration of the status of physical
    parameters traditionally regarded as fundamental.
    If mass and charge are effective descriptors tied to descriptive saturation, their role in physical
    theories may be understood as contingent rather than primitive.

    More broadly, the present work illustrates how structural limitations of effective descriptions can
    shape the form of physical laws.
    Exploring similar mechanisms in other domains may provide insight into the emergence of physical
    properties without invoking additional fundamental entities.

  \subsection{Concluding remarks}
    \label{subsec:conclusion}

    We have proposed a unified structural interpretation of mass, electric charge, and inertia as
    effective manifestations of saturated response under non-injective projection.
    The framework is fully consistent with Standard Model phenomenology and does not modify its
    dynamical content.

    By shifting attention from microscopic dynamics to descriptive structure, the analysis offers a
    coherent conceptual account of why these quantities play the role they do in physical theories.
    While deliberately modest in its claims, the framework provides a basis for further exploration of
    the structural origin of physical properties.
