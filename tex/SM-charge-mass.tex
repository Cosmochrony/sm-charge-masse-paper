\documentclass[pdflatex,sn-mathphys-num]{sn-jnl} % Math and Physical Sciences Numbered Reference Style
\usepackage{amsmath,amssymb,amsfonts}
\usepackage{graphicx}
\usepackage{hyperref}
\usepackage{physics}
\usepackage{tikz}
\usepackage{amsopn}
\usepackage{amstext}
\usepackage{microtype}
\usepackage{xparse}
\usepackage{algorithm}
\usepackage{algorithmicx}
\usepackage{algpseudocode}
% Boxes
\usepackage[most]{tcolorbox}
% Plots (pgfplots provides axis/addplot)
\usepackage{pgfplots}
\usetikzlibrary{intersections}
\usepgfplotslibrary{fillbetween}
\pgfplotsset{compat=1.18}
\NewDocumentCommand{\subsubsubsection}{s m}{%
  \IfBooleanTF{#1}{\paragraph*{#2}}{\paragraph{#2}}%
}
\usetikzlibrary
{arrows.meta, calc, fit, decorations.pathmorphing, decorations.pathreplacing, shapes.geometric, shadings, positioning,
  decorations.markings, patterns}
\theoremstyle{remark}
\newtheorem*{remark}{Remark}

\title[Cosmochrony]
{Charge, Mass, and Inertia as Saturated Responses in a Pre-Geometric Relational Description}
\author*[1]{\fnm{Jérôme} \sur{Beau}}\email{jerome.beau@cosmochrony.org}
\affil*[1]{\orgname{Independent Researcher}, \country{France}}
\date{}

\abstract{
  We investigate the structural origin of mass, electric charge, and inertial
  response within a pre-geometric relational framework in which physical
  observables arise through a generally non-injective projection onto effective
  spacetime descriptions.
  In this setting, geometric and dynamical quantities are not fundamental but
  emerge as regime-dependent responses of an underlying relational description
  subject to intrinsic saturation bounds.

  We show that mass and charge can be consistently interpreted as distinct symmetry
  realizations of a single bounded-response mechanism.
  Mass corresponds to an isotropic inhibition of relational relaxation, while
  electric charge arises as an oriented saturation of the same underlying flux.
  Inertia emerges as the linear-response limit of a bounded projective update
  process, rather than as an intrinsic resistance to motion.

  In addition to the conceptual analysis, we provide an explicit effective
  realization of the saturation mechanism in Lagrangian form.
  A finite projective capacity naturally induces Born--Infeld--type nonlinearities,
  from which mass-like spectral gaps, bounded inertial response, and local $U(1)$
  charge structure arise as symmetry consequences of saturation.

  The proposed framework does not modify Standard Model dynamics and does not
  introduce additional fundamental fields.
  Standard Model phenomenology is recovered as the unsaturated, linear-response
  regime of the effective description, while finite saturation scales imply
  necessary high-gradient limits.

  The results suggest that mass, charge, and inertia represent complementary
  limits of a single saturated relational mechanism, providing a unified and
  falsifiable structural perspective on inertial, gravitational, and
  electromagnetic response.
}

\keywords
{Pre-geometric frameworks; emergent spacetime; mass and charge; inertia; bounded response; non-injective projection;
relational description; saturation mechanisms; Born–Infeld-type dynamics; foundations of physics}

\begin{document}
  \maketitle

  \section{Introduction}
    \label{sec:introduction}

    The physical notions of mass, electric charge, and inertia occupy a central role
    in modern physics.
    They enter as fundamental parameters in both classical and quantum theories,
    yet their conceptual origin remains largely unexplained.

    In the Standard Model, mass and charge are introduced as intrinsic properties of
    elementary fields, with numerical values fixed by empirical input.
    While the Higgs mechanism provides a dynamical account of mass generation for
    gauge bosons and fermions, it does not address why mass exists as a property, nor
    why inertial response accompanies it~\cite{Higgs1964}.
    Electric charge, similarly, is treated as a primitive coupling constant,
    constrained by gauge symmetry but not derived from a deeper structural
    principle~\cite{PeskinSchroeder}.

    From a foundational perspective, inertia presents an additional conceptual
    challenge.
    The resistance of a system to acceleration is usually taken as a defining
    feature of matter, yet its relation to gravitational mass and its possible
    emergence from more primitive structures have been the subject of long-standing
    debate~\cite{Mach1893,Einstein1916}.

    A parallel line of inquiry has emerged in approaches to quantum gravity and
    pre-geometric physics.
    In these frameworks, spacetime geometry, and sometimes even locality and
    causality, are not assumed as fundamental, but are reconstructed as effective
    descriptions from more primitive relational or algebraic
    structures~\cite{Rovelli2004,AmelinoCamelia2013}.
    Within such approaches, familiar physical quantities may acquire an emergent or
    regime-dependent status.

    Recent work has emphasized that effective physical descriptions often arise
    through projections from an underlying configuration space to observable
    spacetime variables.
    When such projections are non-injective, multiple underlying configurations
    correspond to the same effective observable state, leading to intrinsic
    information loss at the descriptive level~\cite{Beau2026b}.
    This structural feature has been shown to account for the breakdown of classical
    probabilistic factorization in quantum correlations, without invoking nonlocal
    dynamics~\cite{Bell1964}.

    In parallel, bounded-response mechanisms have been extensively studied in
    effective field theories.
    Born--Infeld electrodynamics provides a paradigmatic example in which divergences
    are regulated by intrinsic saturation of the field
    response~\cite{BornInfeld1934}.
    Related saturation phenomena appear in gravitational and cosmological contexts,
    where effective responses cease to scale linearly beyond certain thresholds.

    Motivated by these developments, we examine whether mass, electric charge, and
    inertial response can be understood as manifestations of a single structural
    mechanism, and whether such an interpretation imposes non-trivial constraints on
    effective physical descriptions.
    Specifically, we consider a pre-geometric relational framework in which physical
    observables arise through a projection subject to intrinsic saturation bounds.

    Within this setting, mass is interpreted as an isotropic inhibition of
    relational relaxation, while electric charge corresponds to an oriented or
    asymmetric saturation of the same underlying relational flux.
    Inertia then emerges as the linear-response limit of a bounded projective update
    process, reflecting the finite resolvability of changes in relational
    configurations rather than a primitive dynamical property.

    The purpose of this work is not to modify the Standard Model or to introduce new
    fundamental fields.
    Instead, it is to identify structural constraints on how familiar physical
    quantities can consistently arise as effective descriptors within a
    projection-limited framework.
    In addition to the conceptual analysis, we provide an explicit effective
    construction of the proposed saturation mechanism in Lagrangian form, showing
    that bounded relational projection naturally induces Born--Infeld--type
    nonlinearities and associated symmetry structures.

    The paper is organized as follows.
    Section~\ref{sec:framework} introduces the relational and projective framework
    underlying the analysis.
    Section~\ref{sec:saturation} identifies the symmetry classes of saturated
    effective responses and shows how mass, electric charge, and inertial behavior
    arise as distinct realizations of a bounded-response mechanism.
    Section~\ref{sec:consistency} examines the consistency of this interpretation with
    Standard Model phenomenology and identifies necessary high-gradient limits.
    Section~\ref{sec:formal_derivation} then provides an explicit effective derivation
    of the saturation mechanism, including the emergence of mass gaps, local $U(1)$
    gauge structure, and inertial response as the linear limit of a bounded update
    process.
    We conclude in Section~\ref{sec:discussion} with a discussion of conceptual
    implications, limitations, and possible directions for further investigation.

  \section{Relational Framework and Projective Description}
    \label{sec:framework}

    In this section, we summarize the minimal relational and projective framework
    required for the present analysis.
    The detailed construction of this framework is developed in companion works,
    including the relational and spectral reconstruction of effective
    geometry~\cite{Beau2026a}, the formal role of non-injective projection in
    effective descriptions~\cite{Beau2026b}, and the emergence of bounded-response
    regimes relevant for mass, charge, and inertia~\cite{Beau2026c}.
    Here we restrict attention to the structural assumptions that constrain how
    mass, charge, and inertia can consistently arise as effective descriptors.

    \subsection{Underlying relational description}
      \label{subsec:relational-description}

      We assume that effective spacetime observables arise from an underlying
      relational description, in which configurations encode admissible correlations
      between abstract degrees of freedom.
      No spacetime manifold, metric, or field content is assumed at this level.

      Explicit realizations of such relational descriptions, including spectral
      reconstructions of effective geometry from correlation structure alone, have
      been developed in companion work~\cite{Beau2026a}.
      The present paper does not rely on the details of those constructions, but
      adopts their structural conclusion: spacetime geometry is an effective
      description rather than a fundamental input.

      The underlying relational description should not be interpreted as a
      hidden-variable completion of existing theories.
      It defines a space of admissible configurations whose effective description is
      intrinsically coarse-grained.

      Configurations at this level are not endowed with local coordinates, temporal
      ordering, or intrinsic geometric meaning.
      These notions arise only when relational configurations admit a stable and
      approximately injective representation.

    \subsection{Projection to effective observables}
      \label{subsec:projection}

      Observable physical quantities are defined through a projection
      \begin{equation}
        \Pi : \Omega \rightarrow \mathcal{O},
      \end{equation}
      where $\Omega$ denotes the space of underlying relational configurations and
      $\mathcal{O}$ the space of effective observables accessible within spacetime
      descriptions.

      A key structural feature of this projection is that it is generally
      non-injective.
      Distinct underlying configurations may correspond to the same effective
      observable state.
      This identification is not an approximation, but an intrinsic property of the
      descriptive mapping.

      The physical consequences of non-injective projection have been analyzed in
      detail in the context of quantum correlations~\cite{Beau2026b}.
      There it was shown that non-injectivity provides a sufficient mechanism for the
      failure of classical probabilistic factorization, without invoking nonlocal
      dynamics or hidden-variable assumptions.

      In the present work, the same projective structure is assumed to govern inertial
      and interaction-related quantities.
      Mass, charge, and inertia are therefore treated as effective descriptors defined
      on equivalence classes of underlying relational configurations.

    \subsection{Saturation and bounded response}
      \label{subsec:saturation}

      A central structural requirement of the present framework is that the projection
      $\Pi$ admits an intrinsic saturation.
      Beyond a certain threshold, additional variations in the underlying relational
      configuration cannot be resolved within the effective description.

      This bounded-response behavior has been studied in detail in companion
      work~\cite{Beau2026c}, where it was shown that saturation naturally leads to
      Born--Infeld-type effective dynamics and to the dynamical selection of stable
      geometric regimes.

      Here, saturation is not introduced as a modification of fundamental dynamics.
      It is interpreted as a limitation of the effective description itself,
      reflecting finite projective resolution.
      As a result, effective responses cannot be extrapolated linearly under
      arbitrarily large relational gradients.

      Operationally, departures from linear behavior are therefore not contingent
      features, but necessary consequences of finite projective capacity.

    \subsection{Effective descriptors and regime dependence}
      \label{subsec:effective-descriptors}

      Physical quantities such as mass, electric charge, and inertia are treated as
      effective descriptors defined within the projected space $\mathcal{O}$.
      They characterize how the effective description responds to variations of the
      underlying relational configuration.

      These quantities are inherently regime dependent.
      In domains where the projection is approximately injective and unsaturated,
      classical descriptions are recovered and effective responses behave linearly.
      In saturated regimes, departures from linearity arise as a direct consequence
      of projective limitations.

      Importantly, the present framework does not modify the operational content of
      the Standard Model in experimentally tested regimes.
      It provides a structural reinterpretation of existing quantities while
      constraining the conditions under which their linear extrapolation remains
      valid.

      Within this perspective, distinctions between mass, charge, and inertial
      response correspond to different symmetry classes of saturated effective
      behavior.
      This classification is developed in the following sections.

  \section{Symmetry Classes of Saturated Effective Responses}
    \label{sec:saturation}

    In this section, we identify the symmetry classes of effective physical behavior
    that arise as stable realizations once saturation of the projective description
    is taken into account.
    The goal is not to introduce new physical entities, but to show that mass,
    electric charge, and inertial response correspond to the only stable symmetry
    realizations of a bounded effective response, without invoking additional
    fundamental degrees of freedom.

    \subsection{Linear and saturated response regimes}
      \label{subsec:linear-saturated}

      When the projection from underlying relational configurations to effective
      observables is approximately injective, variations at the underlying level are
      faithfully reflected in the effective description.
      In this regime, effective responses scale linearly with the magnitude of
      relational gradients.

      Such linear behavior underlies the standard formulation of classical field
      theories.
      Small perturbations lead to proportionate responses, and superposition
      principles apply.

      However, when relational gradients exceed the resolving capacity of the
      effective description, the projection necessarily enters a saturated regime.
      Beyond this threshold, additional variations at the underlying level no longer
      produce distinct effective outcomes.

      This transition from linear to saturated response is structural rather than
      dynamical.
      It reflects a limitation of descriptive resolution, not a modification of
      underlying laws.
      Bounded-response regimes of this type are known to give rise to Born--Infeld-like
      effective dynamics in a variety of contexts~\cite{BornInfeld1934,Beau2026c}.

    \subsection{Isotropic saturation and effective mass}
      \label{subsec:isotropic}

      We first consider saturated responses that preserve isotropy in the effective
      description.
      In this case, saturation suppresses relational variations uniformly in all
      directions.

      An isotropic inhibition of effective response leads to behavior
      characteristic of inertial mass and to the symmetry properties underlying
      gravitational response.
      The effective description resists changes in motion independently of direction,
      yielding a scalar parameter that quantifies the degree of saturation.

      From this perspective, mass is not introduced as an intrinsic substance.
      It appears as a measure of how strongly the effective description inhibits
      relational reconfiguration once saturation is reached.

      This interpretation is consistent with the equivalence between inertial and
      gravitational mass.
      Both correspond to isotropic limitations of effective response, differing only
      in the context in which the response is probed.
      The universality of free fall follows as a direct consequence of the symmetry of
      the saturation mechanism.

    \subsection{Oriented saturation and effective charge}
      \label{subsec:oriented}

      We now consider saturated responses that break isotropy while preserving
      locality and stability of the effective description.
      In this class, saturation is directionally biased with respect to relational
      gradients.

      Such oriented saturation necessarily leads to behavior characteristic of
      electric charge.
      The effective response distinguishes between opposing directions, giving rise
      to attractive and repulsive interactions depending on orientation.

      Charge thus appears as a signed quantity associated with asymmetric saturation
      of relational flux.
      The existence of both positive and negative charges reflects the presence of two
      stable orientations of the saturated response.

      Importantly, this interpretation does not require introducing charge as a
      primitive coupling.
      It arises as a symmetry-breaking mode of the same bounded-response mechanism
      that gives rise to mass.

      The long-range character of electromagnetic interactions follows from the fact
      that oriented saturation does not induce isotropic suppression, allowing
      extended field-like behavior within the effective description.

    \subsection{Inertia as finite-resolution response}
      \label{subsec:inertia}

      Inertial behavior emerges when changes in motion probe the finite update
      capacity of the effective description.
      Acceleration corresponds to a demand for rapid reconfiguration of relational
      correlations within the projected space.

      When such reconfiguration remains well below the projective capacity, the
      effective response is linear.
      In this regime, resistance to acceleration scales proportionally with the
      applied stress, reproducing the standard inertial relation as a linear-response
      approximation.

      As the required rate of reconfiguration approaches the resolving limit of the
      projection, the effective response saturates.
      Beyond this point, additional relational variation cannot be resolved
      instantaneously, and the projected dynamics exhibits a bounded update behavior.

      In this view, inertia is not a fundamental property attached to matter.
      It is the linear-response limit of a bounded projective update process,
      reflecting the finite resolvability of variations in relational configurations
      within the effective description.

      This interpretation preserves all empirical content of relativistic kinematics
      while reinterpreting inertial resistance as a consequence of finite descriptive
      capacity rather than a primitive dynamical postulate.

    \subsection{Unified classification}
      \label{subsec:classification}

      The analysis above yields a classification of the \emph{minimal} effective physical responses
      permitted by a bounded projective description under a single scalar saturation constraint.

      Within this scope, three stable response classes are singled out.
      Isotropic saturated responses correspond to mass-like behavior.
      Oriented saturated responses correspond to charge-like behavior.
      Finite-resolution resistance to reconfiguration corresponds to inertia.

      All three arise from the same structural mechanism: saturation of the effective description
      under non-injective projection.
      Their distinction follows from symmetry properties of the saturation manifold rather than
      from distinct underlying substances or independent fundamental fields.

      This classification establishes existence rather than completeness.
      It shows that, once bounded projective capacity is assumed, mass-like, charge-like, and
      inertial responses necessarily appear as the minimal symmetry classes compatible with a
      scalar saturation bound.
      It does not yet claim to exhaust all possible gauge structures.

      In particular, partially anisotropic or higher-rank saturation patterns are not excluded.
      Such patterns correspond to saturation manifolds with higher-dimensional internal
      degeneracies.
      For an $N$-component relational multiplet subject to a scalar bound
      $\rho^2=\vec{\phi}\cdot\vec{\phi}$, the symmetry group preserving saturation is $O(N)$, whose
      connected component $SO(N)$ becomes non-abelian for $N\geq 3$.
      Local compatibility of such degeneracies naturally leads to non-abelian gauge redundancies.

      Accordingly, the present work identifies the minimal rank-one realization of bounded-response
      symmetry classes and delineates a concrete extension path toward higher-rank oriented
      saturation.
      The embedding of the full Standard Model gauge group is deferred to a dedicated analysis of
      multi-channel projection and stability conditions.

      In the following section, we examine the consistency of this framework with established
      phenomenology and show that bounded-response symmetry classes imply necessary limits on
      linear extrapolations.

  \section{Consistency with Standard Model Phenomenology}
    \label{sec:consistency}

    In this section, we examine the compatibility of the proposed structural
    interpretation with established phenomenology of the Standard Model.
    The purpose is not to derive Standard Model parameters, but to ensure that
    the framework does not conflict with known experimental facts or operational
    definitions.
    We further identify necessary physical consequences that must follow if the
    saturated-response interpretation is meaningful.

    \subsection{Status of mass in the Standard Model}
      \label{subsec:sm-mass}

      In the Standard Model, particle masses arise through the Higgs mechanism,
      which endows fermions and gauge bosons with effective mass terms via
      spontaneous symmetry breaking.
      This mechanism successfully accounts for the dynamical generation of mass
      and its role in particle interactions.

      The present framework does not challenge this description.
      Instead, it addresses a logically distinct question: why mass appears as a
      universal scalar measure of inertial response and of the coupling to gravitational
      interaction.

      Within the proposed interpretation, the Higgs mechanism determines how mass
      values are assigned within the effective description, while the structural
      origin of mass as an isotropic saturated response explains why such a
      parameter has the physical meaning it does.
      The two perspectives operate at different conceptual levels and are therefore
      complementary rather than competing.

    \subsection{Electric charge and gauge structure}
      \label{subsec:sm-charge}

      Electric charge in the Standard Model is associated with local gauge
      invariance and conserved currents.
      Its quantization and coupling structure are fixed by the underlying gauge
      symmetry and anomaly cancellation requirements.

      The framework developed here does not modify gauge symmetry or charge
      conservation.
      It reinterprets electric charge as an effective manifestation of oriented
      saturation in the projective response, without altering its operational role.

      From this perspective, gauge invariance constrains how oriented saturation can
      appear consistently within the effective description.
      The existence of discrete charge values reflects the stability of specific
      oriented saturation modes, rather than an arbitrary assignment of coupling
      constants.

      Importantly, this interpretation does not predict deviations from known
      electromagnetic phenomena in the weak-field regime.
      All standard results of quantum electrodynamics are recovered in the
      unsaturated and approximately injective limit.

    \subsection{Inertia, relativistic dynamics, and equivalence}
      \label{subsec:inertia-consistency}

      Relativistic dynamics treats inertia as a fundamental response encoded in the
      energy--momentum relation.
      The equivalence between inertial and gravitational mass is experimentally
      well established and constitutes a cornerstone of relativistic physics.

      Within the present framework, this equivalence arises naturally from the
      isotropy of saturated response.
      Both inertial resistance to acceleration and gravitational response correspond
      to the same structural limitation of effective reconfiguration.
      Inertia itself is shown to emerge as the linear-response limit of a bounded
      projective update process, while gravitational response probes the same
      saturation mechanism through spacetime geometry.

      No modification of relativistic kinematics is implied in the unsaturated regime.
      Lorentz invariance remains an effective symmetry of the projected description
      as long as the projection is approximately injective.
      Lorentz invariance remains an effective symmetry of the projected
      description, valid in regimes where the projection is approximately injective
      and unsaturated.

    \subsection{Absence of observable deviations at accessible scales}
      \label{subsec:absence}

      A crucial consistency requirement is the absence of observable deviations
      from Standard Model predictions in experimentally tested regimes.
      The framework satisfies this requirement by construction.

      Saturation effects become relevant only when relational gradients approach the
      resolving capacity of the effective description.
      In present-day particle physics experiments, electromagnetic and inertial
      responses remain well within the linear regime, and no saturation is probed.

      As a result, the framework predicts no departures from established cross
      sections, decay rates, or precision tests of quantum electrodynamics and
      electroweak theory.
      All Standard Model phenomenology is recovered as the linear-response limit of
      the effective description.

    \subsection{High-energy consistency and saturation limits}
      \label{subsec:high-energy-consistency}

      The absence of deviations at accessible scales does not imply that effective
      responses can be extrapolated linearly to arbitrarily high gradients.
      If mass, charge, and inertia arise as saturated responses constrained by a
      finite projective bound $b$, then strictly unbounded growth of effective
      couplings would render the framework internally inconsistent.

      The consistency of the saturated-response interpretation therefore requires
      the existence of regimes in which perturbative extrapolations must fail.
      Such deviations are not additional assumptions or phenomenological
      predictions, but necessary consequences of introducing a finite projective
      capacity.

      An empirical indication that such bounded behavior already exists is provided
      by the well-established Schwinger limit, which sets an upper threshold on
      sustainable electromagnetic field invariants before vacuum instability
      occurs.
      In the present framework, this threshold is interpreted as the operational
      manifestation of a finite projective bandwidth of the vacuum.

      We emphasize that the present work does not address the microscopic dynamics
      of vacuum instability or pair production.
      The detailed realization of saturation effects in strong-field quantum
      electrodynamics will be developed in a companion paper.
      Here, the Schwinger limit serves solely as an existence proof that unbounded
      extrapolation of effective responses is physically untenable.

      This interpretation is explicitly falsifiable.
      If effective couplings can be shown to grow without bound under arbitrarily
      strong field gradients, with no indication of saturation or inflection, then
      the bounded-response framework developed here must be rejected.

    \subsection{Relation to other bounded-response frameworks}
      \label{subsec:bounded}

      Bounded-response mechanisms have previously been considered in both
      electromagnetic and gravitational contexts.
      Born--Infeld electrodynamics provides a well-known example in which saturation
      regulates divergent field strengths without spoiling low-energy
      phenomenology~\cite{BornInfeld1934}.

      The present framework generalizes this idea conceptually.
      Rather than introducing bounded response as a modification of specific field
      equations, saturation is interpreted as a generic limitation of effective
      descriptions arising from non-injective projection.

      This shift in perspective allows mass, charge, and inertia to be treated on
      equal footing, as different symmetry realizations of the same structural
      mechanism.

    \subsection{Summary}
      \label{subsec:consistency-summary}

      The proposed interpretation is fully consistent with Standard Model
      phenomenology.
      It does not alter gauge structure, particle content, or dynamical equations in
      experimentally tested regimes.

      At the same time, it identifies necessary high-gradient limits in which
      linear extrapolations of effective responses must break down.
      Standard Model physics is recovered as the unsaturated regime of a bounded
      projective description.

      In the following section, we discuss conceptual implications, limitations, and
      possible directions for further investigation.

  \section{Formal Derivation of Saturated Responses}
    \label{sec:formal_derivation}

    This section adds the minimal mathematical structure required to connect the relational
    framework to effective physical response.
    It provides an explicit effective action for the projected relaxation degree of freedom and
    shows how mass-like, charge-like, and inertial responses arise under finite projective
    capacity.
    The goal is not to postulate new fundamental fields, but to identify the minimal effective
    structures necessarily induced by bounded relational update flux.

    \paragraph{Scope of Section 5.}
      This section provides a formal effective derivation of mass-like, charge-like, and inertial
      responses from bounded projective capacity.
      The objective is to establish existence and internal consistency rather than microscopic
      uniqueness.
      Completeness with respect to the full Standard Model gauge structure is not claimed and is
      deferred to higher-rank oriented saturation analyses.
      Throughout this section, the term ``derivation'' is used in this effective sense, with all
      closure assumptions stated explicitly.

  \subsection{Relational flux and projective capacity}
  \label{subsec:relational_flux}

  We represent the local relational update content by a rank-two tensor density
  \begin{equation}
    J_{\mu\nu} \equiv \partial_{\mu}\phi \, \partial_{\nu}\phi,
    \label{eq:relational_tensor}
  \end{equation}
  where $\phi$ encodes the effective relaxation mode surviving projection.
  The projection $\Pi$ is assumed to have finite capacity.
  Operationally, this means that the observable description cannot resolve arbitrarily large
  values of relational flux invariants constructed from $J_{\mu\nu}$.
  We parametrize this limitation by introducing a saturation scale $b$ with dimensions of a
  flux density.

  The minimal Lorentz-invariant completion enforcing boundedness while recovering the standard
  quadratic kinetic term at low flux is the Dirac--Born--Infeld-type Lagrangian
  \begin{equation}
    \mathcal{L}_{\mathrm{BI}}(\phi)
    \equiv b^2\left(1-\sqrt{1-\frac{\partial_{\mu}\phi \,
    \partial^{\mu}\phi}{b^2}}\right).
    \label{eq:lagrangian_dbi_scalar}
  \end{equation}
  Equivalently, the associated action reads
  \begin{equation}
    S_{\Pi}[\phi]
    = -b^2 \int d^4x
    \left(
      \sqrt{-\det\left(
                   \eta_{\mu\nu}-b^{-2}\partial_{\mu}\phi \,
                   \partial_{\nu}\phi
      \right)} - 1
    \right),
    \label{eq:action_bi}
  \end{equation}
  making explicit that saturation corresponds to an effective deformation of the metric
  volume element as relational flux approaches $b$.

  For weak gradients, $\partial\phi \ll b$, Eq.~\eqref{eq:lagrangian_dbi_scalar} expands as
  \begin{equation}
    \mathcal{L}_{\mathrm{BI}}(\phi)
    = \frac{1}{2}\partial_{\mu}\phi \, \partial^{\mu}\phi
    + \frac{1}{8b^2}
    \left(\partial_{\mu}\phi \, \partial^{\mu}\phi\right)^2
    + \mathcal{O}\!\left(b^{-4}\right),
    \label{eq:dbi_expansion}
  \end{equation}
  recovering the linear-response regime.

  \subsection{Isotropic saturation and the emergence of a mass scale}
  \label{subsec:isotropic_mass}

  Isotropic saturation corresponds to saturation reached through the scalar invariant
  \begin{equation}
    X \equiv \partial_{\mu}\phi \, \partial^{\mu}\phi,
    \label{eq:kinetic_invariant}
  \end{equation}
  rather than through direction-dependent components.

  Equation~\eqref{eq:lagrangian_dbi_scalar} enforces boundedness but does not generate a mass
  gap.
  A gap arises when isotropic saturation selects a typical finite-flux background.
  In a projection-limited regime, $X$ aggregates many unresolved relational configurations.
  Under a fixed capacity bound, coarse-graining produces concentration of measure around a
  typical saturation value $X_{\star}$.

  We encode this statistical rigidity through the effective completion
  \begin{equation}
    \mathcal{L}_{\mathrm{iso}}
    \equiv \mathcal{L}_{\mathrm{BI}}(\phi)
    - \frac{\kappa}{2}\left(X-X_{\star}\right)^2,
    \label{eq:iso_completion}
  \end{equation}
  where $X_{\star}$ denotes the saturation-typical value.
  We stress that the quadratic form is not claimed to be uniquely derived.
  The existence of a typical value follows from concentration arguments, but the detailed
  functional form of the penalty reflects a minimal local and analytic closure.
  Alternative closures are not excluded.
  The role of Eq.~\eqref{eq:iso_completion} is to demonstrate the existence of a mass gap
  under isotropic saturation, not to provide a microscopic derivation of the penalty.

  Let $\phi=\phi_0+\delta\phi$ with $X[\phi_0]=X_{\star}$.
  Expanding to second order yields
  \begin{equation}
    \mathcal{L}_{\delta\phi}
    \simeq
    \frac{1}{2}Z(X_{\star},b)\,
    \partial_{\mu}\delta\phi\,\partial^{\mu}\delta\phi
    - \frac{1}{2}\kappa\,(\delta\phi)^2,
  \end{equation}
  with
  \begin{equation}
    Z(X_{\star},b)
    \equiv \left(1-\frac{X_{\star}}{b^2}\right)^{-3/2}.
  \end{equation}
  The equation of motion takes the Klein--Gordon form
  \begin{equation}
    \left(\Box + m_{\mathrm{eff}}^2\right)\delta\phi = 0,
    \label{eq:kg_effective}
  \end{equation}
  where
  \begin{equation}
    m_{\mathrm{eff}}^2 = \frac{\kappa}{Z(X_{\star},b)}.
    \label{eq:meff_general}
  \end{equation}

  \subsection{Inertia as the linear limit of saturated update rates}
  \label{subsec:inertia_update}

  To relate bounded DBI kinetics to inertial response, we consider a particle-like reduction.
  Specifically, we focus on homogeneous or slowly varying modes for which the dominant effect
  of saturation is captured by a single collective coordinate $x(t)$.
  This reduction is valid when spatial gradients are subdominant and the observable probes
  global reconfiguration rates rather than local field structure.

  For such a mode, the DBI kinetic sector induces the generalized momentum
  \begin{equation}
    p = \frac{m_{\mathrm{eff}}\,\dot{x}}{\sqrt{1-\dot{x}^2/a_{\star}^2}},
  \end{equation}
  where $a_{\star}$ denotes the maximal admissible update rate.
  Defining $\mathcal{F}\equiv dp/dt$, the equation of motion becomes
  \begin{equation}
    \ddot{x}
    = \frac{\mathcal{F}}{m_{\mathrm{eff}}}
    \left[
      1+\left(\frac{\mathcal{F}}{m_{\mathrm{eff}}a_{\star}}\right)^2
    \right]^{-1/2}.
    \label{eq:dynamical_derivation}
  \end{equation}
  This equation follows from the Legendre transform of the DBI kinetic term and exhibits a
  linear regime together with asymptotic saturation.

  The phenomenological form
  \begin{equation}
    \frac{d^2 \mathcal{O}}{dt^2}
    = a_{\star}\,
    \mathcal{S}\!\left(
                   \frac{\mathcal{F}}{m_{\mathrm{eff}} a_{\star}}
    \right),
    \label{eq:saturated_fma}
  \end{equation}
  with $\mathcal{S}(x)=\tanh(x)$, represents a smooth analytic closure of the same universality
  class.
  Inertia thus emerges as the linear-response limit of bounded update dynamics.

  \subsection{Oriented saturation and the emergence of a local $U(1)$ symmetry}
  \label{subsec:oriented_u1}

  If at least two relational degrees of freedom remain unresolved, the saturation-preserving
  manifold is a compact connected one-dimensional space, uniquely $S^1$.
  A minimal realization is
  \begin{equation}
    \psi \equiv \rho\,e^{i\theta},
    \label{eq:complex_mode}
  \end{equation}
  with $\rho$ fixed near $\rho_{\star}$ and $\theta$ underdetermined.
  The redundancy
  \begin{equation}
    \theta(x)\rightarrow\theta(x)+\alpha(x)
    \label{eq:local_phase}
  \end{equation}
  requires a compensating connection.
  Introducing
  \begin{equation}
    D_{\mu}\psi \equiv (\partial_{\mu}-iqA_{\mu})\psi,
    \label{eq:covariant_derivative}
  \end{equation}
  yields
  \begin{equation}
    \mathcal{L}_{\mathrm{ori}}
    = b^2\left(
           1-\sqrt{1-\frac{D_{\mu}\psi\,D^{\mu}\psi^{\ast}}{b^2}}
    \right)
    - \frac{1}{4}F_{\mu\nu}F^{\mu\nu}.
    \label{eq:oriented_lagrangian}
  \end{equation}

  \subsection{Immediate falsifiability}
  \label{subsec:falsifiability_bound}

  The effective action~\eqref{eq:action_bi} ceases to define a real description when
  \begin{equation}
    \partial_{\mu}\phi\,\partial^{\mu}\phi > b^2.
    \label{eq:bound_violation}
  \end{equation}
  The framework is ruled out if strictly linear behavior is experimentally required beyond
  this scale.

  Universality of $b$ provides a second falsifiability channel.
  Identifying the saturation threshold with the Schwinger field
  $E_{\mathrm{S}}\sim b/e$ yields the benchmark
  \begin{equation}
    b \sim 10^{24}\,\mathrm{J\,m^{-3}},
  \end{equation}
  up to order-unity factors.
  Failure of cross-sector consistency between inertial, mass, and strong-field phenomena
  would falsify the bounded-response hypothesis.

  \section{Discussion, Limitations, and Outlook}
  \label{sec:discussion}

  In this final section, we summarize the conceptual implications of the proposed
  framework, clarify its limitations, and outline possible directions for future
  investigation.
  While the analysis remains primarily structural, it now includes an explicit
  effective realization of the saturation mechanism.
  This allows physical constraints to be formulated at the level of an effective
  Lagrangian, rather than at the level of interpretation alone.

  \subsection{Conceptual implications}
    \label{subsec:implications}

    A central implication of the present analysis is that mass, electric charge, and
    inertia need not be treated as fundamentally distinct physical primitives.
    Instead, they can be consistently understood as different symmetry realizations
    of a single structural mechanism, namely saturation of effective response under
    non-injective projection.

    From this perspective, the apparent diversity of physical properties reflects
    differences in how the effective description responds to relational variation,
    rather than differences in underlying substance.
    Scalar, oriented, and finite-bandwidth response modes correspond respectively to
    mass-like, charge-like, and inertial behavior, with inertia arising as the
    linear-response limit of a bounded projective update process.

    The explicit effective construction introduced in
    Section~\ref{sec:formal_derivation} shows that these distinctions can be realized
    within a single bounded-response framework, without modifying the operational
    definitions of mass and charge.
    In this sense, the analysis establishes existence and internal consistency of
    these response classes rather than a unique microscopic derivation.
    Mass appears as a spectral gap induced by isotropic saturation, while electric
    charge emerges as the generator of a local phase degeneracy associated with
    oriented saturation.
    Inertia, in turn, corresponds to the first-order expansion of a saturated update
    dynamics.

    This interpretation provides a unified conceptual basis for the equivalence
    between inertial and gravitational mass and for the signed nature of electric
    charge, while preserving their empirical roles within established physical
    theories.

  \subsection{Relation to existing foundational approaches}
    \label{subsec:relation}

    The framework developed here is compatible with a wide range of foundational
    approaches in which spacetime and physical observables are regarded as effective
    constructs.
    It does not rely on a specific microscopic ontology and can be embedded in
    different relational, background-independent, or pre-geometric settings.

    Unlike approaches that postulate new degrees of freedom or modified fundamental
    dynamics, the present work operates deliberately at the level of effective
    description.
    Its contribution is to isolate a minimal structural mechanism that constrains
    how familiar physical notions can consistently arise within effective theories
    subject to finite descriptive capacity.

    In this sense, the framework complements rather than replaces existing
    formulations.
    It provides an interpretative and constraining layer that may coexist with
    standard quantum field theory and relativistic dynamics, while clarifying the
    domain of validity of linear-response extrapolations.

  \subsection{Limitations of the present work}
    \label{subsec:limitations}

    The present analysis is subject to several important limitations.

    First, no microscopic dynamics of the underlying relational description is
    specified.
    As a result, the framework does not claim to derive unique functional forms for
    all effective closures, nor to predict numerical values for particle masses,
    charges, or coupling constants.
    These remain empirical inputs determined within effective theories operating in
    the unsaturated regime.

    Second, the specific form of the effective penalty associated with isotropic
    saturation is introduced as a minimal analytic closure.
    While the existence of a typical saturation value follows from concentration
    arguments, the detailed functional form of the penalty is not uniquely fixed at
    this level of description and is deferred to future work.

    Third, the analysis does not provide detailed quantitative predictions for
    high-energy or strong-field deviations from Standard Model behavior.
    Nevertheless, the effective formalism developed in
    Section~\ref{sec:formal_derivation} identifies a well-defined departure from
    strictly linear inertial response and introduces characteristic saturation scales,
    implying necessary limits on linear extrapolations.

    Finally, the framework does not address particle generations, flavor structure,
    or symmetry-breaking patterns within the Standard Model.
    Its scope is restricted to the structural origin and effective interpretation of
    mass, charge, and inertia.

  \subsection{Possible extensions and outlook}
    \label{subsec:outlook}

    Despite these limitations, the framework suggests several concrete directions for
    further investigation.

    One natural extension is the construction of explicit relational or spectral
    models in which the projective bound and its associated saturation scale can be
    derived rather than postulated.
    Such developments would allow the present effective closures to be replaced by
    microscopically grounded expressions.

    Another important direction concerns the systematic study of strong-field and
    high-gradient regimes, where saturation effects are expected to become
    operational.
    Clarifying the relation between bounded-response mechanisms and known nonlinear
    phenomena in quantum field theory may help identify experimentally accessible
    signatures.

    More broadly, the analysis invites reconsideration of the status of physical
    parameters traditionally regarded as fundamental.
    If mass and charge are effective descriptors tied to finite descriptive capacity,
    their role in physical theories may be understood as constrained rather than
    primitive.

  \subsection{Concluding remarks}
    \label{subsec:conclusion}

    We have proposed a unified structural interpretation of mass, electric charge, and
    inertia as effective manifestations of saturated response under non-injective
    projection.
    The framework is fully consistent with Standard Model phenomenology and does not
    modify its dynamical content in experimentally tested regimes.

    By providing an explicit effective realization of the saturation mechanism, the
    present work goes beyond purely interpretative unification and introduces
    well-defined structural constraints.
    In particular, the emergence of inertia as the linear limit of a bounded update
    process implies necessary high-gradient limits, rendering the framework
    falsifiable in principle.

    While deliberately modest in its quantitative claims, the analysis establishes a
    coherent and internally consistent basis for further exploration of the
    structural origin of physical properties.

  \appendix

    \backmatter

    \bmhead{Acknowledgements}
    The author acknowledges the use of large language models as a supportive tool
    for refining language, structure, and internal consistency during the
    development of this manuscript.
    All conceptual contributions, theoretical choices, and interpretations remain the sole responsibility of the author.

    \bibliography{references}

\end{document}
