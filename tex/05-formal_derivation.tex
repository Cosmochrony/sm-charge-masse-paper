\section{Formal Derivation of Saturated Responses}
  \label{sec:formal_derivation}

  This section adds the minimal mathematical structure missing from the current manuscript.
  It provides an explicit effective action for the projected relaxation degree of freedom and
  derives mass-like, charge-like, and inertial responses as consequences of a finite projective
  capacity.
  The goal is not to postulate new fundamental fields, but to demonstrate that a bounded
  relational update flux necessarily induces a Born--Infeld-type effective completion in the
  observable description.

  \paragraph{Scope of Section 5.}
    This section provides a minimal effective derivation of mass-like, charge-like, and inertial
    responses from bounded projective capacity.
    The goal is to establish existence and internal consistency of the saturated-response
    mechanism.
    Completeness with respect to the full Standard Model gauge structure is not claimed at this
    stage, and is deferred to higher-rank oriented saturation analyses.
    Whenever a step is presented as an effective closure rather than a microscopic derivation, we
    state it explicitly and indicate the corresponding derivation program.

  \subsection{Relational flux and projective capacity}
    \label{subsec:relational_flux}

    We represent the local relational update content by a rank-two tensor density
    \begin{equation}
      J_{\mu\nu} \equiv \partial_{\mu}\phi \, \partial_{\nu}\phi,
      \label{eq:relational_tensor}
    \end{equation}
    where $\phi$ is an effective scalar encoding the relaxation mode that remains after
    projection.
    The essential assumption is that the projection $\Pi$ has a finite capacity.
    Operationally, this means that the observable description cannot resolve arbitrarily large
    values of relational flux invariants constructed from $J_{\mu\nu}$.
    We encode this limitation by introducing a saturation scale $b$ with dimensions of a flux
    density.

    The minimal Lorentz-invariant completion that enforces boundedness while recovering the
    standard quadratic kinetic term at low flux is the Dirac--Born--Infeld-type Lagrangian
    \begin{equation}
      \mathcal{L}_{\mathrm{BI}}(\phi)
      \equiv b^2\left(1-\sqrt{1-\frac{\partial_{\mu}\phi \,
      \partial^{\mu}\phi}{b^2}}\right).
      \label{eq:lagrangian_dbi_scalar}
    \end{equation}
    Equivalently, the associated action can be written as
    \begin{equation}
      S_{\Pi}[\phi]
      = \int d^4x \, \mathcal{L}_{\mathrm{BI}}(\phi)
      = -b^2 \int d^4x
      \left(
        \sqrt{-\det\left(
                     \eta_{\mu\nu}-b^{-2}\partial_{\mu}\phi \,
                     \partial_{\nu}\phi
        \right)} - 1
      \right),
      \label{eq:action_bi}
    \end{equation}
    where the determinant form makes explicit that saturation is controlled by an effective
    stretching of the metric volume element as the relational flux approaches $b$.

    For weak gradients, $\partial\phi \ll b$, Eq.~\eqref{eq:lagrangian_dbi_scalar} expands as
    \begin{equation}
      \mathcal{L}_{\mathrm{BI}}(\phi)
      = \frac{1}{2}\partial_{\mu}\phi \, \partial^{\mu}\phi
      + \frac{1}{8b^2}
      \left(\partial_{\mu}\phi \, \partial^{\mu}\phi\right)^2
      + \mathcal{O}\!\left(b^{-4}\right),
      \label{eq:dbi_expansion}
    \end{equation}
    so that the unsaturated regime reproduces the standard linear-response kinetic term.

  \subsection{Isotropic saturation and the emergence of a mass scale}
    \label{subsec:isotropic_mass}

    We now demonstrate how an effective mass scale arises when saturation is isotropic.
    Isotropy here means that saturation is reached through a scalar invariant of the flux,
    \begin{equation}
      X \equiv \partial_{\mu}\phi \, \partial^{\mu}\phi,
      \label{eq:kinetic_invariant}
    \end{equation}
    rather than through direction-dependent tensor components.

    Equation~\eqref{eq:lagrangian_dbi_scalar} alone enforces boundedness but does not generate a
    mass gap.
    A mass scale appears when isotropic saturation selects a typical finite-flux background
    around which fluctuations are defined.
    In a projection-limited regime, the observable invariant $X$ summarizes many unresolved
    relational configurations.
    Under a fixed capacity bound, coarse-graining generically produces concentration of measure
    around a typical saturation value.
    The effective equilibrium is therefore not at $X=0$, but at a saturation-typical stasis level
    $X_{\star}$.

    We encode this statistical rigidity by introducing the minimal local penalty on deviations
    from the typical saturated value,
    \begin{equation}
      \mathcal{L}_{\mathrm{iso}}
      \equiv \mathcal{L}_{\mathrm{BI}}(\phi)
      - \frac{\kappa}{2}\left(X-X_{\star}\right)^2,
      \label{eq:iso_completion}
    \end{equation}
    where $X_{\star}$ is not a new fundamental parameter.
    It represents the typical saturation level selected by a maximum-entropy principle under the
    capacity constraint.
    The coefficient $\kappa>0$ quantifies the residual bandwidth, namely the variance of admissible
    saturated microstates after projection.
    This term does not introduce an independent interaction channel, but encodes the elastic
    response of the projection once saturation has been reached.

    Let $\phi=\phi_0+\delta\phi$, with $X[\phi_0]=X_{\star}$.
    Linearizing the Euler--Lagrange equation for $\delta\phi$ yields a Klein--Gordon form
    \begin{equation}
      \left(\Box + m_{\mathrm{eff}}^2\right)\delta\phi = 0,
      \label{eq:kg_effective}
    \end{equation}
    with an effective mass gap
    \begin{equation}
      m_{\mathrm{eff}}^2
      = 2\kappa \, X_{\star}
      \left.
        \left(\frac{\partial X}{\partial \phi}\right)^2
      \right|_{\phi_0}
      \times \mathcal{Z}^{-1}(X_{\star},b),
      \label{eq:meff_general}
    \end{equation}
    where $\mathcal{Z}(X_{\star},b)$ denotes the DBI renormalization of the kinetic term evaluated
    on the saturated background.
    As $X_{\star}\rightarrow b^2$, $\mathcal{Z}$ diverges, reflecting the progressive loss of
    sensitivity of the projected dynamics to further increases of relational flux.
    The resulting low-energy spectrum exhibits a finite gap with dispersion relation
    \begin{equation}
      E^2 = p^2 c^2 + m_{\mathrm{eff}}^2 c^4.
      \label{eq:dispersion_gap}
    \end{equation}

    An isotropically saturated projected description therefore admits a mass-like descriptor
    because saturation induces an elastic strain with a finite energetic cost for local
    reconfiguration.
    Mass appears as the effective label of a persistent isotropic projective strain.

  \subsection{Inertia as the linear limit of saturated update rates}
    \label{subsec:inertia_update}

    We now demonstrate that the second law of motion emerges as the linear-response limit of a
    bounded projective update process.
    The DBI structure implies bounded canonical momenta.
    For a homogeneous mode, the time-like kinetic sector reads
    \begin{equation}
      \mathcal{L}_{\mathrm{BI}}
      = b^2\left(1-\sqrt{1-\dot{\phi}^2/b^2}\right),
    \end{equation}
    yielding the conjugate momentum
    \begin{equation}
      p_{\phi}
      = \frac{\partial \mathcal{L}_{\mathrm{BI}}}{\partial \dot{\phi}}
      = \frac{\dot{\phi}}{\sqrt{1-\dot{\phi}^2/b^2}}.
    \end{equation}
    As $|\dot{\phi}|\rightarrow b$, the momentum cost diverges, so increasingly strong forcing
    cannot indefinitely increase the effective update rate.

    At the level of an effective observable $\mathcal{O}(t)$, this bounded-update property is
    captured by a saturated response law
    \begin{equation}
      \frac{d^2 \mathcal{O}}{dt^2}
      = a_{\star}\,
      \mathcal{S}\!\left(
                     \frac{\mathcal{F}}{m_{\mathrm{eff}} a_{\star}}
      \right),
      \label{eq:saturated_fma}
    \end{equation}
    where $\mathcal{S}$ is a smooth odd saturation function.
    The choice $\mathcal{S}(x)=\tanh(x)$ is a convenient representative.
    The structural prediction is the existence of a finite asymptotic acceleration.

    In the low-stress regime, Eq.~\eqref{eq:saturated_fma} reduces to
    \begin{equation}
      \frac{d^2 \mathcal{O}}{dt^2}
      \approx \frac{\mathcal{F}}{m_{\mathrm{eff}}},
    \end{equation}
    recovering the Newtonian relation as the first-order expansion of a bounded-response law.
    Inertia thus appears as the linear scaling factor of a finite projective bandwidth.

    \subsection{Oriented saturation and the emergence of a local $U(1)$ symmetry}
    \label{subsec:oriented_u1}

    We next consider the oriented class, in which saturation preserves a scalar bound while
    leaving an internal degeneracy.
    If at least two relational degrees of freedom remain unresolved, the manifold of saturation-
    preserving configurations is a compact connected one-dimensional space.
    Up to homeomorphism, the unique such manifold without boundary is the circle $S^1$.

    A minimal realization is therefore a two-component field
    $\psi \equiv \phi_1+i\phi_2=\rho e^{i\theta}$,
    \begin{equation}
      \psi \equiv \rho \, e^{i\theta}.
      \label{eq:complex_mode}
    \end{equation}
    Saturation fixes $\rho$ near $\rho_{\star}$, while the phase $\theta$ remains locally
    underdetermined.
    The projected description is invariant under
    \begin{equation}
      \theta(x)\rightarrow \theta(x)+\alpha(x),
      \label{eq:local_phase}
    \end{equation}
    which expresses a redundancy rather than a physical transformation.

    Local redundancy requires a compensating connection.
    Introducing
    \begin{equation}
      D_{\mu}\psi \equiv \left(\partial_{\mu}-iqA_{\mu}\right)\psi,
      \label{eq:covariant_derivative}
    \end{equation}
    yields the minimal effective Lagrangian
    \begin{equation}
      \mathcal{L}_{\mathrm{ori}}
      = b^2\left(
             1-\sqrt{1-\frac{D_{\mu}\psi \, D^{\mu}\psi^{\ast}}{b^2}}
      \right)
      - \frac{1}{4}F_{\mu\nu}F^{\mu\nu},
      \label{eq:oriented_lagrangian}
    \end{equation}
    with $F_{\mu\nu}\equiv \partial_{\mu}A_{\nu}-\partial_{\nu}A_{\mu}$.
    The group generated by Eq.~\eqref{eq:local_phase} is $U(1)$.
    Electric charge labels the Noether generator associated with this oriented saturation
    redundancy.

    \subsection{Immediate falsifiability}
    \label{subsec:falsifiability_bound}

    The effective action~\eqref{eq:action_bi} ceases to define a real projected description when
    the invariant exceeds the saturation bound,
    \begin{equation}
      \partial_{\mu}\phi \, \partial^{\mu}\phi > b^2.
      \label{eq:bound_violation}
    \end{equation}
    The framework is therefore ruled out if a strictly linear unsaturated continuation is
    experimentally required beyond the scale $b$ without any sign of instability, saturation, or
    nonlinear completion.

    A second falsifiability channel is internal consistency across sectors.
    If the same saturation scale $b$ underlies mass generation and oriented gauge response, then
    the values of $b$ inferred from inertial, mass, and strong-field charge phenomena cannot be
    chosen independently.
    Failure of this cross-sector consistency falsifies the unified bounded-response hypothesis.
