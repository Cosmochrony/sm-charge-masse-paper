\section{Formal Derivation of Saturated Responses}
  \label{sec:formal_derivation}

  This section adds the minimal mathematical structure missing from the current manuscript.
  It provides an explicit effective action for the projected relaxation degree of freedom and
  establishes mass-like, charge-like, and inertial responses as consequences of a finite
  projective capacity.
  The goal is not to postulate new fundamental fields, but to show that a bounded relational
  update flux necessarily induces a Born--Infeld-type effective completion in the observable
  description.

  \paragraph{Scope of Section 5.}
    This section provides a formal effective derivation of mass-like, charge-like, and inertial
    responses from bounded projective capacity.
    The goal is to establish existence and internal consistency of the saturated-response
    mechanism.
    Completeness with respect to the full Standard Model gauge structure is not claimed at this
    stage and is deferred to higher-rank oriented saturation analyses.
    Throughout this section, the term ``derivation'' is used in this effective sense rather than
    as a microscopic reconstruction.

\subsection{Relational flux and projective capacity}
\label{subsec:relational_flux}

We represent the local relational update content by a rank-two tensor density
\begin{equation}
  J_{\mu\nu} \equiv \partial_{\mu}\phi \, \partial_{\nu}\phi,
  \label{eq:relational_tensor}
\end{equation}
where $\phi$ is an effective scalar encoding the relaxation mode that remains after
projection.
The essential assumption is that the projection $\Pi$ has a finite capacity.
Operationally, this means that the observable description cannot resolve arbitrarily large
values of relational flux invariants constructed from $J_{\mu\nu}$.
We encode this limitation by introducing a saturation scale $b$ with dimensions of a flux
density.

The minimal Lorentz-invariant completion enforcing boundedness while recovering the standard
quadratic kinetic term at low flux is the Dirac--Born--Infeld-type Lagrangian
\begin{equation}
  \mathcal{L}_{\mathrm{BI}}(\phi)
  \equiv b^2\left(1-\sqrt{1-\frac{\partial_{\mu}\phi \,
  \partial^{\mu}\phi}{b^2}}\right).
  \label{eq:lagrangian_dbi_scalar}
\end{equation}
Equivalently, the associated action reads
\begin{equation}
  S_{\Pi}[\phi]
  = -b^2 \int d^4x
  \left(
    \sqrt{-\det\left(
                 \eta_{\mu\nu}-b^{-2}\partial_{\mu}\phi \,
                 \partial_{\nu}\phi
    \right)} - 1
  \right),
  \label{eq:action_bi}
\end{equation}
making explicit that saturation is controlled by an effective deformation of the metric
volume element as relational flux approaches $b$.

For weak gradients, $\partial\phi \ll b$, Eq.~\eqref{eq:lagrangian_dbi_scalar} expands as
\begin{equation}
  \mathcal{L}_{\mathrm{BI}}(\phi)
  = \frac{1}{2}\partial_{\mu}\phi \, \partial^{\mu}\phi
  + \frac{1}{8b^2}
  \left(\partial_{\mu}\phi \, \partial^{\mu}\phi\right)^2
  + \mathcal{O}\!\left(b^{-4}\right),
  \label{eq:dbi_expansion}
\end{equation}
recovering the linear-response regime.

\subsection{Isotropic saturation and the emergence of a mass scale}
\label{subsec:isotropic_mass}

We now show how an effective mass scale arises when saturation is isotropic.
Isotropy means that saturation is reached through the scalar invariant
\begin{equation}
  X \equiv \partial_{\mu}\phi \, \partial^{\mu}\phi,
  \label{eq:kinetic_invariant}
\end{equation}
rather than through direction-dependent components.

Equation~\eqref{eq:lagrangian_dbi_scalar} alone enforces boundedness but does not generate a
mass gap.
A gap appears when isotropic saturation selects a typical finite-flux background.
In a projection-limited regime, $X$ summarizes many unresolved relational configurations.
Under a fixed capacity bound, coarse-graining produces concentration of measure around a
typical saturation value $X_{\star}$.

We encode this statistical rigidity through the minimal completion
\begin{equation}
  \mathcal{L}_{\mathrm{iso}}
  \equiv \mathcal{L}_{\mathrm{BI}}(\phi)
  - \frac{\kappa}{2}\left(X-X_{\star}\right)^2,
  \label{eq:iso_completion}
\end{equation}
where $X_{\star}$ is not fundamental but selected by a maximum-entropy principle.
The coefficient $\kappa>0$ measures the residual bandwidth of admissible saturated
microstates.

Let $\phi=\phi_0+\delta\phi$ with $X[\phi_0]=X_{\star}$.
Expanding to second order yields
\begin{equation}
  \mathcal{L}_{\delta\phi}
  \simeq
  \frac{1}{2}Z(X_{\star},b)\,
  \partial_{\mu}\delta\phi\,\partial^{\mu}\delta\phi
  - \frac{1}{2}\kappa\,(\delta\phi)^2,
\end{equation}
with
\begin{equation}
  Z(X_{\star},b)
  \equiv \left(1-\frac{X_{\star}}{b^2}\right)^{-3/2}.
\end{equation}
The equation of motion takes the Klein--Gordon form
\begin{equation}
  \left(\Box + m_{\mathrm{eff}}^2\right)\delta\phi = 0,
  \label{eq:kg_effective}
\end{equation}
where
\begin{equation}
  m_{\mathrm{eff}}^2 = \frac{\kappa}{Z(X_{\star},b)}.
  \label{eq:meff_general}
\end{equation}
The mass gap thus results from the combined effect of relational stiffness $\kappa$ and
projective impedance $Z$ induced by proximity to saturation.

\subsection{Inertia as the linear limit of saturated update rates}
\label{subsec:inertia_update}

We now establish inertial response from the equations of motion of the Born--Infeld
structure.
For a particle-like reduction with effective coordinate $x(t)$, the DBI kinetic sector
induces a generalized momentum
\begin{equation}
  p = \frac{m_{\mathrm{eff}}\,\dot{x}}{\sqrt{1-\dot{x}^2/a_{\star}^2}},
\end{equation}
where $a_{\star}$ is the maximal admissible update rate.

Defining the effective force as $\mathcal{F}\equiv dp/dt$, the equation of motion becomes
\begin{equation}
  \ddot{x}
  = \frac{\mathcal{F}}{m_{\mathrm{eff}}}
  \left[
    1+\left(\frac{\mathcal{F}}{m_{\mathrm{eff}}a_{\star}}\right)^2
  \right]^{-1/2}.
  \label{eq:dynamical_derivation}
\end{equation}
This equation follows directly from the Legendre transform of the DBI kinetic term.
It exhibits a linear regime $\mathcal{F}\approx m_{\mathrm{eff}}a$ and an asymptotic
saturation $|\ddot{x}|\to a_{\star}$.

The phenomenological form
\begin{equation}
  \frac{d^2 \mathcal{O}}{dt^2}
  = a_{\star}\,
  \mathcal{S}\!\left(
                 \frac{\mathcal{F}}{m_{\mathrm{eff}} a_{\star}}
  \right),
  \label{eq:saturated_fma}
\end{equation}
with $\mathcal{S}(x)=\tanh(x)$, represents a smooth analytic closure of the same
universality class.
Inertia therefore emerges as the linear-response limit of a bounded update dynamics
derived from the DBI structure.

\subsection{Oriented saturation and the emergence of a local $U(1)$ symmetry}
\label{subsec:oriented_u1}

If at least two relational degrees of freedom remain unresolved, the saturation-preserving
manifold is a compact connected one-dimensional space, uniquely $S^1$.
A minimal realization is
\begin{equation}
  \psi \equiv \rho\,e^{i\theta},
  \label{eq:complex_mode}
\end{equation}
with $\rho$ fixed near $\rho_{\star}$ and $\theta$ underdetermined.
The redundancy
\begin{equation}
  \theta(x)\rightarrow\theta(x)+\alpha(x)
  \label{eq:local_phase}
\end{equation}
requires a compensating connection.
Introducing
\begin{equation}
  D_{\mu}\psi \equiv (\partial_{\mu}-iqA_{\mu})\psi,
  \label{eq:covariant_derivative}
\end{equation}
yields
\begin{equation}
  \mathcal{L}_{\mathrm{ori}}
  = b^2\left(
         1-\sqrt{1-\frac{D_{\mu}\psi\,D^{\mu}\psi^{\ast}}{b^2}}
  \right)
  - \frac{1}{4}F_{\mu\nu}F^{\mu\nu}.
  \label{eq:oriented_lagrangian}
\end{equation}
The associated symmetry group is $U(1)$.
Electric charge labels the Noether generator of this oriented saturation redundancy.

\subsection{Immediate falsifiability}
\label{subsec:falsifiability_bound}

The effective action~\eqref{eq:action_bi} ceases to define a real description when
\begin{equation}
  \partial_{\mu}\phi\,\partial^{\mu}\phi > b^2.
  \label{eq:bound_violation}
\end{equation}
The framework is ruled out if strictly linear behavior is experimentally required beyond
this scale.

Universality of $b$ provides a second falsifiability channel.
Identifying the saturation threshold with the Schwinger field
$E_{\mathrm{S}}\sim b/e$ yields the benchmark
\begin{equation}
  b \sim 10^{24}\,\mathrm{J\,m^{-3}},
\end{equation}
up to order-unity factors.
Failure of cross-sector consistency between inertial, mass, and strong-field phenomena
would falsify the bounded-response hypothesis.
