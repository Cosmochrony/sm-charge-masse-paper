\section{Formal Derivation of Saturated Responses}
  \label{sec:formal_derivation}

  This section adds the minimal mathematical structure missing from the current manuscript.
  It provides an explicit effective action for the projected relaxation degree of freedom and derives the
  mass-like and charge-like symmetry classes as consequences of a finite projective capacity.
  The goal is not to postulate new fundamental fields, but to show that a bounded relational update flux
  forces a Born--Infeld-type effective completion in the observable description~:contentReference[oaicite:0]{index=0}.

  \subsection{Relational flux and projective capacity}
    \label{subsec:relational_flux}

    We represent the local relational ``update'' content by a rank-two tensor density
    \begin{equation}
      J_{\mu\nu} \equiv \partial_{\mu}\phi \, \partial_{\nu}\phi,
      \label{eq:relational_tensor}
    \end{equation}
    where $\phi$ is an effective scalar encoding the relaxation mode that remains after projection.
    The essential assumption is that the projection $\Pi$ has a finite capacity.
    Operationally, this means that the observable description cannot resolve arbitrarily large values of the
    relational flux invariants constructed from $J_{\mu\nu}$.
    We encode this limitation by introducing a saturation scale $b$ with dimensions of a flux density.

    The minimal Lorentz-invariant completion that enforces boundedness while recovering the standard
    quadratic kinetic term at low flux is the Dirac--Born--Infeld-type Lagrangian
    \begin{equation}
      \mathcal{L}_{\mathrm{BI}}(\phi)
      \equiv b^2\left(1-\sqrt{1-\frac{\partial_{\mu}\phi \, \partial^{\mu}\phi}{b^2}}\right).
      \label{eq:lagrangian_dbi_scalar}
    \end{equation}
    Equivalently, one may write the associated action as
    \begin{equation}
      S_{\Pi}[\phi]
      = \int d^4x \, \mathcal{L}_{\mathrm{BI}}(\phi)
      = -b^2 \int d^4x
      \left(\sqrt{-\det\left(\eta_{\mu\nu}-b^{-2}\partial_{\mu}\phi \, \partial_{\nu}\phi\right)}-1\right),
      \label{eq:action_bi}
    \end{equation}
    where the determinant form makes explicit that saturation is controlled by an effective stretching of
    the metric volume element as the flux approaches $b$.

    For weak gradients, $\partial\phi \ll b$, Eq.~\eqref{eq:lagrangian_dbi_scalar} expands as
    \begin{equation}
      \mathcal{L}_{\mathrm{BI}}(\phi)
      = \frac{1}{2}\partial_{\mu}\phi \, \partial^{\mu}\phi
      + \frac{1}{8b^2}\left(\partial_{\mu}\phi \, \partial^{\mu}\phi\right)^2
      + \mathcal{O}\!\left(b^{-4}\right),
      \label{eq:dbi_expansion}
    \end{equation}
    so that the unsaturated regime reproduces the standard linear-response kinetic term, with controlled
    nonlinear corrections suppressed by $b$.

  \subsection{Isotropic saturation and the emergence of a mass scale}
    \label{subsec:isotropic_mass}

    We now show how an effective mass scale arises when saturation is isotropic.
    Isotropy here means that saturation is reached through a scalar invariant of the flux,
    \begin{equation}
      X \equiv \partial_{\mu}\phi \, \partial^{\mu}\phi,
      \label{eq:kinetic_invariant}
    \end{equation}
    rather than through direction-dependent tensor components.

    A key point is that Eq.~\eqref{eq:lagrangian_dbi_scalar} by itself does not generate a mass term.
    A mass scale appears when isotropic saturation locks the system near a finite-flux background.
    We model this lock by requiring that the projected description maintains a preferred saturation level
    $X_{\star}$ through a weak restoring term that represents the cost of reconfiguring an already
    saturated relational neighborhood.
    The minimal effective completion is
    \begin{equation}
      \mathcal{L}_{\mathrm{iso}}
      \equiv \mathcal{L}_{\mathrm{BI}}(\phi)
      - \frac{\kappa}{2}\left(X-X_{\star}\right)^2,
      \label{eq:iso_completion}
    \end{equation}
    where $\kappa>0$ encodes the stiffness of the saturation constraint in the projected description.
    This term does not introduce a new interaction channel.
    It encodes, at the level of $\mathcal{O}$, the fact that once projection saturates, departures from the
    saturated configuration carry an energetic penalty.

    Let $\phi=\phi_0+\delta\phi$, with $\phi_0$ such that $X[\phi_0]=X_{\star}$.
    Linearizing the Euler--Lagrange equation for $\delta\phi$ yields a Klein--Gordon form
    \begin{equation}
      \left(\Box + m_{\mathrm{eff}}^2\right)\delta\phi = 0,
      \label{eq:kg_effective}
    \end{equation}
    with an effective gap
    \begin{equation}
      m_{\mathrm{eff}}^2
      = 2\kappa \, X_{\star}
      \left.\left(\frac{\partial X}{\partial \phi}\right)^2\right|_{\phi_0}
      \times \mathcal{Z}^{-1}(X_{\star},b),
      \label{eq:meff_general}
    \end{equation}
    where $\mathcal{Z}(X_{\star},b)$ is the DBI renormalization of the kinetic term evaluated on the
    saturated background.
    In particular, $\mathcal{Z}$ diverges as $X_{\star}\rightarrow b^2$, reflecting the fact that the
    projected dynamics becomes progressively insensitive to further increases of flux.
    This produces a stable low-energy spectrum with a finite gap.
    The corresponding relativistic dispersion relation reads
    \begin{equation}
      E^2 = p^2 c^2 + m_{\mathrm{eff}}^2 c^4.
      \label{eq:dispersion_gap}
    \end{equation}

    Equation~\eqref{eq:dispersion_gap} provides the sought mathematical statement.
    An isotropically saturated projected description admits a mass-like descriptor because saturation
    creates a background strain with a finite energetic cost for local reconfiguration.
    In this interpretation, mass is not a primitive substance.
    Mass is the effective label of a persistent isotropic projective strain.

  \subsection{Oriented saturation and the emergence of a local $U(1)$ symmetry}
    \label{subsec:oriented_u1}

    We next consider the oriented class, in which saturation preserves a scalar bound while allowing an
    internal phase degeneracy.
    We therefore introduce a complex relaxation mode
    \begin{equation}
      \psi \equiv \rho \, e^{i\theta},
      \label{eq:complex_mode}
    \end{equation}
    and impose that saturation fixes the modulus $\rho$ near a preferred value $\rho_{\star}$ while leaving
    the phase $\theta$ as a locally underdetermined coordinate of the projection.
    This is precisely the situation in which the projected description becomes insensitive to an internal
    rotation.
    The minimal statement of this insensitivity is invariance under
    \begin{equation}
      \theta(x)\rightarrow \theta(x)+\alpha(x).
      \label{eq:local_phase}
    \end{equation}

    Local invariance~\eqref{eq:local_phase} forces the appearance of a compensating connection.
    One introduces a gauge field $A_{\mu}$ and the covariant derivative
    \begin{equation}
      D_{\mu}\psi \equiv \left(\partial_{\mu}-iqA_{\mu}\right)\psi,
      \label{eq:covariant_derivative}
    \end{equation}
    so that $D_{\mu}\psi$ transforms covariantly under $\psi\rightarrow e^{i\alpha(x)}\psi$.
    The effective Lagrangian for the oriented saturated sector then takes the minimal form
    \begin{equation}
      \mathcal{L}_{\mathrm{ori}}
      = b^2\left(1-\sqrt{1-\frac{D_{\mu}\psi \, D^{\mu}\psi^{\ast}}{b^2}}\right)
      - \frac{1}{4}F_{\mu\nu}F^{\mu\nu},
      \label{eq:oriented_lagrangian}
    \end{equation}
    with $F_{\mu\nu}\equiv \partial_{\mu}A_{\nu}-\partial_{\nu}A_{\mu}$.

    The group implied by Eq.~\eqref{eq:local_phase} is $U(1)$, because the internal degeneracy is a single
    compact phase degree of freedom preserving a scalar saturation bound.
    In this construction, charge is the Noether generator associated with phase rotations in the saturated
    flux sector.
    Charge is therefore not postulated as an intrinsic property.
    Charge labels the oriented symmetry class of a bounded projective response.

  \subsection{Inertia as projective drag and finite observational bandwidth}
    \label{subsec:inertia_drag}

    Finally, we connect inertial response to a finite update bandwidth of the projection.
    Let $B_{\Pi}$ denote the maximal rate at which the projected description can update observables.
    For an observable $\mathcal{O}$, the minimal constraint reads
    \begin{equation}
      \left\lVert \frac{d\mathcal{O}}{dt} \right\rVert \leq B_{\Pi}.
      \label{eq:bandwidth}
    \end{equation}
    Equivalently, changes of the projection satisfy the scaling
    \begin{equation}
      \Delta\Pi \sim \frac{1}{b}\frac{d\mathcal{O}}{dt},
      \label{eq:delta_pi_drag}
    \end{equation}
    where $b$ plays the role of a flux normalization converting an attempted rapid change of the effective
    state into a finite projective strain.

    Inertial behavior arises when external forcing demands an update rate approaching the bound.
    In that regime, the projected dynamics cannot follow instantaneously.
    The resulting lag acts as an effective resistance to acceleration, consistent with the interpretation
    of inertia as a response limited by finite projective capacity rather than as a primitive force.

  \subsection{Immediate falsifiability}
    \label{subsec:falsifiability_bound}

    The formalism above yields a direct falsifiability criterion.
    The effective action~\eqref{eq:action_bi} ceases to be real when the invariant exceeds the bound,
    \begin{equation}
      \partial_{\mu}\phi \, \partial^{\mu}\phi > b^2,
      \label{eq:bound_violation}
    \end{equation}
    so the framework is ruled out if an experimentally well-defined regime requires a strictly linear,
    unsaturated continuation beyond the scale $b$ without any sign of instability, saturation, or
    nonlinear completion.
    In the next revision, this criterion should be anchored to a physical lower bound on $b$ by tying it to
    a known strong-field threshold, providing an operational test of the bounded-response hypothesis.
