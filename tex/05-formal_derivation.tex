\section{Formal Derivation of Saturated Responses}
  \label{sec:formal_derivation}

  This section adds the minimal mathematical structure missing from the current manuscript.
  It provides an explicit effective action for the projected relaxation degree of freedom and
  derives mass-like and charge-like symmetry classes as consequences of a finite projective
  capacity.
  The goal is not to postulate new fundamental fields, but to demonstrate that a bounded
  relational update flux necessarily induces a Born--Infeld-type effective completion in the
  observable description.

  \subsection{Relational flux and projective capacity}
    \label{subsec:relational_flux}

    We represent the local relational update content by a rank-two tensor density
    \begin{equation}
      J_{\mu\nu} \equiv \partial_{\mu}\phi \, \partial_{\nu}\phi,
      \label{eq:relational_tensor}
    \end{equation}
    where $\phi$ is an effective scalar encoding the relaxation mode that remains after
    projection.
    The essential assumption is that the projection $\Pi$ has a finite capacity.
    Operationally, this means that the observable description cannot resolve arbitrarily large
    values of relational flux invariants constructed from $J_{\mu\nu}$.
    We encode this limitation by introducing a saturation scale $b$ with dimensions of a flux
    density.

    The minimal Lorentz-invariant completion that enforces boundedness while recovering the
    standard quadratic kinetic term at low flux is the Dirac--Born--Infeld-type Lagrangian
    \begin{equation}
      \mathcal{L}_{\mathrm{BI}}(\phi)
      \equiv b^2\left(1-\sqrt{1-\frac{\partial_{\mu}\phi \,
      \partial^{\mu}\phi}{b^2}}\right).
      \label{eq:lagrangian_dbi_scalar}
    \end{equation}
    Equivalently, the associated action can be written as
    \begin{equation}
      S_{\Pi}[\phi]
      = \int d^4x \, \mathcal{L}_{\mathrm{BI}}(\phi)
      = -b^2 \int d^4x
      \left(
        \sqrt{-\det\left(
                     \eta_{\mu\nu}-b^{-2}\partial_{\mu}\phi \,
                     \partial_{\nu}\phi
        \right)} - 1
      \right),
      \label{eq:action_bi}
    \end{equation}
    where the determinant form makes explicit that saturation is controlled by an effective
    stretching of the metric volume element as the relational flux approaches $b$.

    For weak gradients, $\partial\phi \ll b$, Eq.~\eqref{eq:lagrangian_dbi_scalar} expands as
    \begin{equation}
      \mathcal{L}_{\mathrm{BI}}(\phi)
      = \frac{1}{2}\partial_{\mu}\phi \, \partial^{\mu}\phi
      + \frac{1}{8b^2}
      \left(\partial_{\mu}\phi \, \partial^{\mu}\phi\right)^2
      + \mathcal{O}\!\left(b^{-4}\right),
      \label{eq:dbi_expansion}
    \end{equation}
    so that the unsaturated regime reproduces the standard linear-response kinetic term,
    with nonlinear corrections suppressed by the saturation scale $b$.

  \subsection{Isotropic saturation and the emergence of a mass scale}
    \label{subsec:isotropic_mass}

    We now demonstrate how an effective mass scale arises when saturation is isotropic.
    Isotropy here means that saturation is reached through a scalar invariant of the flux,
    \begin{equation}
      X \equiv \partial_{\mu}\phi \, \partial^{\mu}\phi,
      \label{eq:kinetic_invariant}
    \end{equation}
    rather than through direction-dependent tensor components.

    Equation~\eqref{eq:lagrangian_dbi_scalar} alone does not generate a mass term.
    A mass scale appears when isotropic saturation locks the system near a finite-flux
    background.
    We model this locking by requiring that the projected description maintains a preferred
    saturation level $X_{\star}$ through a weak restoring term representing the elastic cost
    of reconfiguring an already saturated relational neighborhood.
    The minimal effective completion reads
    \begin{equation}
      \mathcal{L}_{\mathrm{iso}}
      \equiv \mathcal{L}_{\mathrm{BI}}(\phi)
      - \frac{\kappa}{2}\left(X-X_{\star}\right)^2,
      \label{eq:iso_completion}
    \end{equation}
    where $\kappa>0$ encodes the stiffness of the saturation constraint in the projected
    description.
    This term does not introduce an independent interaction channel.
    It encodes the elastic response of the projection once saturation has been reached.

    Let $\phi=\phi_0+\delta\phi$, with $X[\phi_0]=X_{\star}$.
    Linearizing the Euler--Lagrange equation for $\delta\phi$ yields a Klein--Gordon form
    \begin{equation}
      \left(\Box + m_{\mathrm{eff}}^2\right)\delta\phi = 0,
      \label{eq:kg_effective}
    \end{equation}
    with an effective mass gap
    \begin{equation}
      m_{\mathrm{eff}}^2
      = 2\kappa \, X_{\star}
      \left.
        \left(\frac{\partial X}{\partial \phi}\right)^2
      \right|_{\phi_0}
      \times \mathcal{Z}^{-1}(X_{\star},b),
      \label{eq:meff_general}
    \end{equation}
    where $\mathcal{Z}(X_{\star},b)$ denotes the DBI renormalization of the kinetic term
    evaluated on the saturated background.
    As $X_{\star}\rightarrow b^2$, $\mathcal{Z}$ diverges, reflecting the progressive loss of
    sensitivity of the projected dynamics to further increases of relational flux.
    The resulting low-energy spectrum exhibits a finite gap with dispersion relation
    \begin{equation}
      E^2 = p^2 c^2 + m_{\mathrm{eff}}^2 c^4.
      \label{eq:dispersion_gap}
    \end{equation}

    Equation~\eqref{eq:dispersion_gap} provides the sought mathematical statement.
    An isotropically saturated projected description admits a mass-like descriptor because
    saturation induces an elastic strain with a finite energetic cost for local
    reconfiguration.
    Mass is therefore not a primitive substance, but the effective label of a persistent
    isotropic projective strain.

  \subsection{Inertia as the linear limit of saturated update rates}
    \label{subsec:inertia_update}

    We now demonstrate that the second law of motion,
    $F = ma$, emerges as the linear-response limit of a bounded projective update
    process.
    In the Cosmochrony framework, a change in the effective state of an observable
    $x \in \mathcal{O}$ requires a reconfiguration of the underlying relational
    substrate $\chi$.
    This reconfiguration is limited by a finite projective bandwidth $B_{\Pi}$,
    which provides a temporal manifestation of the saturation bound $b$.

    Let $\mathcal{O}(t)$ denote the effective description of a physical system.
    The rate at which this description can be updated in response to a relational
    stress $\mathcal{F}$ is therefore bounded.
    At the effective level, this constraint can be modeled by a saturated response
    equation of the form
    \begin{equation}
      \frac{d^2 \mathcal{O}}{dt^2}
      = a_{\star}\,
      \mathcal{S}\!\left(
                     \frac{\mathcal{F}}{m_{\mathrm{eff}} a_{\star}}
      \right),
      \label{eq:saturated_fma}
    \end{equation}
    where $m_{\mathrm{eff}} \sim b/c^2$ is the effective mass scale derived in
    Section~\ref{subsec:isotropic_mass},
    $a_{\star}$ is the characteristic acceleration associated with saturation,
    and $\mathcal{S}$ is a smooth odd saturation function satisfying
    $\mathcal{S}(x)\rightarrow x$ for $|x|\ll 1$ and
    $|\mathcal{S}(x)|\rightarrow 1$ for $|x|\gg 1$.
    A convenient representative choice is $\mathcal{S}(x)=\tanh(x)$, though the
    specific functional form is not essential for the argument.

    In the low-stress regime, $\mathcal{F} \ll m_{\mathrm{eff}} a_{\star}$,
    Eq.~\eqref{eq:saturated_fma} reduces to
    \begin{equation}
      \frac{d^2 \mathcal{O}}{dt^2}
      \approx a_{\star}
      \left(
        \frac{\mathcal{F}}{m_{\mathrm{eff}} a_{\star}}
      \right)
      = \frac{\mathcal{F}}{m_{\mathrm{eff}}},
    \end{equation}
    which recovers the Newtonian relation $F = m_{\mathrm{eff}} a$ as the first-order
    expansion of the saturated response.
    Inertia therefore appears not as a fundamental resistance to motion, but as the
    linear scaling factor of a bounded projective update process.

    Beyond the linear regime, when the required acceleration approaches
    $a_{\star}$, the response saturates.
    This reflects the impossibility of resolving arbitrarily rapid changes of the
    relational configuration within a finite projective bandwidth.
    In this interpretation, inertia is the temporal lag of the projection $\Pi$:
    it measures the finite time required for the effective description to track a
    change in the underlying relational structure.

    The existence of a characteristic acceleration scale $a_{\star}$ thus follows
    directly from the bounded-response hypothesis.
    In low-density or weakly constrained environments, this scale can dominate the
    effective dynamics, naturally producing deviations from linear inertia while
    preserving the Newtonian limit at high accelerations.

  \subsection{Oriented saturation and the emergence of a local $U(1)$ symmetry}
    \label{subsec:oriented_u1}

    We next consider the oriented class, in which saturation preserves a scalar bound while
    allowing an internal phase degeneracy.
    We introduce a complex relaxation mode
    \begin{equation}
      \psi \equiv \rho \, e^{i\theta},
      \label{eq:complex_mode}
    \end{equation}
    and impose that saturation fixes the modulus $\rho$ near a preferred value
    $\rho_{\star}$, while leaving the phase $\theta$ as a locally underdetermined coordinate
    of the projection.
    The projected description is therefore insensitive to local internal rotations,
    expressed by the invariance
    \begin{equation}
      \theta(x)\rightarrow \theta(x)+\alpha(x).
      \label{eq:local_phase}
    \end{equation}

    Local invariance~\eqref{eq:local_phase} necessitates the introduction of a compensating
    connection.
    We define the covariant derivative
    \begin{equation}
      D_{\mu}\psi \equiv \left(\partial_{\mu}-iqA_{\mu}\right)\psi,
      \label{eq:covariant_derivative}
    \end{equation}
    so that $D_{\mu}\psi$ transforms covariantly under
    $\psi\rightarrow e^{i\alpha(x)}\psi$.
    The effective Lagrangian for the oriented saturated sector takes the minimal form
    \begin{equation}
      \mathcal{L}_{\mathrm{ori}}
      = b^2\left(
             1-\sqrt{1-\frac{D_{\mu}\psi \, D^{\mu}\psi^{\ast}}{b^2}}
      \right)
      - \frac{1}{4}F_{\mu\nu}F^{\mu\nu},
      \label{eq:oriented_lagrangian}
    \end{equation}
    where $F_{\mu\nu}\equiv \partial_{\mu}A_{\nu}-\partial_{\nu}A_{\mu}$.

    The group implied by Eq.~\eqref{eq:local_phase} is $U(1)$, as it is the unique continuous
    one-parameter compact symmetry preserving a scalar saturation bound.
    In this construction, electric charge is the Noether generator associated with phase
    rotations in the oriented saturated flux sector.
    Charge is therefore not postulated as an intrinsic property, but labels a symmetry class
    of bounded projective response.

    This dynamical formulation admits a direct interpretation in terms of finite projective bandwidth.

    Finally, we connect inertial response to a finite update bandwidth of the projection.
    Let $B_{\Pi}$ denote the maximal rate at which the projected description can update
    observables.
    For an observable $\mathcal{O}$, the minimal constraint reads
    \begin{equation}
      \left\lVert \frac{d\mathcal{O}}{dt} \right\rVert \leq B_{\Pi}.
      \label{eq:bandwidth}
    \end{equation}
    Equivalently, changes of the projection satisfy the scaling
    \begin{equation}
      \Delta\Pi \sim \frac{1}{b}\frac{d\mathcal{O}}{dt},
      \label{eq:delta_pi_drag}
    \end{equation}
    where $b$ acts as a flux normalization converting an attempted rapid change of the
    effective state into a finite projective strain.

    Inertial behavior arises when external forcing demands an update rate approaching the
    bound.
    In this regime, the projected dynamics cannot follow instantaneously.
    The resulting lag manifests as an effective resistance to acceleration, consistent with
    the interpretation of inertia as a response limited by finite projective capacity rather
    than as a primitive force.

  \subsection{Immediate falsifiability}
    \label{subsec:falsifiability_bound}

    The formalism above yields a direct falsifiability criterion.
    The effective action~\eqref{eq:action_bi} ceases to define a real projected description
    when the invariant exceeds the saturation bound,
    \begin{equation}
      \partial_{\mu}\phi \, \partial^{\mu}\phi > b^2,
      \label{eq:bound_violation}
    \end{equation}
    so the framework is ruled out if an experimentally established regime requires a
    strictly linear, unsaturated continuation beyond the scale $b$ without any sign of
    instability, saturation, or nonlinear completion.
    Anchoring this bound to known strong-field thresholds provides a direct operational test
    of the bounded-response hypothesis.
