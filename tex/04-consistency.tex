\section{Consistency with Standard Model Phenomenology}
  \label{sec:consistency}

  In this section, we examine the compatibility of the proposed structural
  interpretation with established phenomenology of the Standard Model.
  The purpose is not to derive Standard Model parameters, but to ensure that
  the framework does not conflict with known experimental facts or operational
  definitions.
  We further identify necessary physical consequences that must follow if the
  saturated-response interpretation is meaningful.

  \subsection{Status of mass in the Standard Model}
    \label{subsec:sm-mass}

    In the Standard Model, particle masses arise through the Higgs mechanism,
    which endows fermions and gauge bosons with effective mass terms via
    spontaneous symmetry breaking.
    This mechanism successfully accounts for the dynamical generation of mass
    and its role in particle interactions.

    The present framework does not challenge this description.
    Instead, it addresses a logically distinct question: why mass appears as a
    universal scalar measure of inertial and gravitational response.

    Within the proposed interpretation, the Higgs mechanism determines how mass
    values are assigned within the effective description, while the structural
    origin of mass as an isotropic saturated response explains why such a
    parameter has the physical meaning it does.
    The two perspectives operate at different conceptual levels and are therefore
    complementary rather than competing.

  \subsection{Electric charge and gauge structure}
    \label{subsec:sm-charge}

    Electric charge in the Standard Model is associated with local gauge
    invariance and conserved currents.
    Its quantization and coupling structure are fixed by the underlying gauge
    symmetry and anomaly cancellation requirements.

    The framework developed here does not modify gauge symmetry or charge
    conservation.
    It reinterprets electric charge as an effective manifestation of oriented
    saturation in the projective response, without altering its operational role.

    From this perspective, gauge invariance constrains how oriented saturation can
    appear consistently within the effective description.
    The existence of discrete charge values reflects the stability of specific
    oriented saturation modes, rather than an arbitrary assignment of coupling
    constants.

    Importantly, this interpretation does not predict deviations from known
    electromagnetic phenomena in the weak-field regime.
    All standard results of quantum electrodynamics are recovered in the
    unsaturated and approximately injective limit.

  \subsection{Inertia, relativistic dynamics, and equivalence}
    \label{subsec:inertia-consistency}

    Relativistic dynamics treats inertia as a fundamental response encoded in the
    energy--momentum relation.
    The equivalence between inertial and gravitational mass is experimentally
    well established and constitutes a cornerstone of relativistic physics.

    Within the present framework, this equivalence arises naturally from the
    isotropy of saturated response.
    Both inertial resistance to acceleration and gravitational response correspond
    to the same structural limitation of effective reconfiguration.
    Inertia itself is shown to emerge as the linear-response limit of a bounded
    projective update process, while gravitational response probes the same
    saturation mechanism through spacetime geometry.

    No modification of relativistic kinematics is implied.
    Lorentz invariance remains an effective symmetry of the projected
    description, valid in regimes where the projection is approximately injective
    and unsaturated.

  \subsection{Absence of observable deviations at accessible scales}
    \label{subsec:absence}

    A crucial consistency requirement is the absence of observable deviations
    from Standard Model predictions in experimentally tested regimes.
    The framework satisfies this requirement by construction.

    Saturation effects become relevant only when relational gradients approach the
    resolving capacity of the effective description.
    In present-day particle physics experiments, electromagnetic and inertial
    responses remain well within the linear regime, and no saturation is probed.

    As a result, the framework predicts no departures from established cross
    sections, decay rates, or precision tests of quantum electrodynamics and
    electroweak theory.
    All Standard Model phenomenology is recovered as the linear-response limit of
    the effective description.

  \subsection{High-energy consistency and saturation limits}
    \label{subsec:high-energy-consistency}

    The absence of deviations at accessible scales does not imply that effective
    responses can be extrapolated linearly to arbitrarily high gradients.
    If mass, charge, and inertia arise as saturated responses constrained by a
    finite projective bound $b$, then strictly unbounded growth of effective
    couplings would render the framework internally inconsistent.

    The consistency of the saturated-response interpretation therefore requires
    the existence of regimes in which perturbative extrapolations must fail.
    Such deviations are not additional assumptions or phenomenological
    predictions, but necessary consequences of introducing a finite projective
    capacity.

    An empirical indication that such bounded behavior already exists is provided
    by the well-established Schwinger limit, which sets an upper threshold on
    sustainable electromagnetic field invariants before vacuum instability
    occurs.
    In the present framework, this threshold is interpreted as the operational
    manifestation of a finite projective bandwidth of the vacuum.

    We emphasize that the present work does not address the microscopic dynamics
    of vacuum instability or pair production.
    The detailed realization of saturation effects in strong-field quantum
    electrodynamics will be developed in a companion paper.
    Here, the Schwinger limit serves solely as an existence proof that unbounded
    extrapolation of effective responses is physically untenable.

    This interpretation is explicitly falsifiable.
    If effective couplings can be shown to grow without bound under arbitrarily
    strong field gradients, with no indication of saturation or inflection, then
    the bounded-response framework developed here must be rejected.

  \subsection{Relation to other bounded-response frameworks}
    \label{subsec:bounded}

    Bounded-response mechanisms have previously been considered in both
    electromagnetic and gravitational contexts.
    Born--Infeld electrodynamics provides a well-known example in which saturation
    regulates divergent field strengths without spoiling low-energy
    phenomenology~\cite{BornInfeld1934}.

    The present framework generalizes this idea conceptually.
    Rather than introducing bounded response as a modification of specific field
    equations, saturation is interpreted as a generic limitation of effective
    descriptions arising from non-injective projection.

    This shift in perspective allows mass, charge, and inertia to be treated on
    equal footing, as different symmetry realizations of the same structural
    mechanism.

  \subsection{Summary}
    \label{subsec:consistency-summary}

    The proposed interpretation is fully consistent with Standard Model
    phenomenology.
    It does not alter gauge structure, particle content, or dynamical equations in
    experimentally tested regimes.

    At the same time, it identifies necessary high-gradient limits in which
    linear extrapolations of effective responses must break down.
    Standard Model physics is recovered as the unsaturated regime of a bounded
    projective description.

    In the following section, we discuss conceptual implications, limitations, and
    possible directions for further investigation.
