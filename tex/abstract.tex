\abstract{
  We investigate the structural origin of mass, electric charge, and inertial response within a pre-geometric
  relational framework in which physical observables arise through a generally non-injective projection
  onto effective spacetime descriptions.
  In this setting, geometric and dynamical quantities are not fundamental but emerge as regime-dependent
  responses of an underlying relational description subject to intrinsic saturation bounds.

  We show that mass and charge can be interpreted as distinct manifestations of the same bounded-response
  mechanism.
  Mass corresponds to an isotropic inhibition of relational relaxation, while electric charge arises as an
  oriented or chiral saturation of the same underlying flux.
  Inertia emerges as a secondary effect, reflecting the finite resolvability of changes in relational
  configurations rather than the presence of an intrinsic inertial property.

  The proposed framework does not modify the Standard Model dynamics and does not introduce additional
  fundamental fields.
  Instead, it provides a structural interpretation of existing quantities as effective descriptors of
  projection-limited responses.
  The familiar distinction between mass and charge is recovered at the effective level, while their
  common origin becomes manifest at the pre-geometric scale.

  We discuss the consistency of this interpretation with known phenomenology, its relation to bounded
  (Born–Infeld-type) effective dynamics, and its implications for unifying inertial, gravitational, and
  electromagnetic responses without invoking new microscopic degrees of freedom.
  The results suggest that mass and charge may be understood as complementary limits of a single saturated
  relational mechanism, offering a unified conceptual basis for inertial and interaction properties.
}
