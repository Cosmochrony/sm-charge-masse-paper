\abstract{
  We investigate the structural origin of mass, electric charge, and inertial
  response within a pre-geometric relational framework in which physical
  observables arise through a generally non-injective projection onto effective
  spacetime descriptions.
  In this setting, geometric and dynamical quantities are not fundamental but
  emerge as regime-dependent responses of an underlying relational description
  subject to intrinsic saturation bounds.

  We show that mass and charge can be consistently interpreted as distinct symmetry
  realizations of a single bounded-response mechanism.
  Mass corresponds to an isotropic inhibition of relational relaxation, while
  electric charge arises as an oriented saturation of the same underlying flux.
  Inertia emerges as the linear-response limit of a bounded projective update
  process, rather than as an intrinsic resistance to motion.

  In addition to the conceptual analysis, we provide an explicit effective
  realization of the saturation mechanism in Lagrangian form.
  A finite projective capacity naturally induces Born--Infeld--type nonlinearities,
  from which mass-like spectral gaps, bounded inertial response, and local $U(1)$
  charge structure arise as symmetry consequences of saturation.

  The proposed framework does not modify Standard Model dynamics and does not
  introduce additional fundamental fields.
  Standard Model phenomenology is recovered as the unsaturated, linear-response
  regime of the effective description, while finite saturation scales imply
  necessary high-gradient limits.

  The results suggest that mass, charge, and inertia represent complementary
  limits of a single saturated relational mechanism, providing a unified and
  falsifiable structural perspective on inertial, gravitational, and
  electromagnetic response.
}
