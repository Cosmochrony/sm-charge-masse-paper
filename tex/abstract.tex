\abstract{
  We investigate the structural origin of mass, electric charge, and inertial
  response within a pre-geometric relational framework in which physical
  observables arise through a generally non-injective projection onto effective
  spacetime descriptions.
  In this setting, geometric and dynamical quantities are not fundamental but
  emerge as regime-dependent responses of an underlying relational description
  subject to intrinsic saturation bounds.

  We show that mass and charge can be interpreted as distinct symmetry
  realizations of a single bounded-response mechanism.
  Mass corresponds to an isotropic inhibition of relational relaxation, while
  electric charge arises as an oriented or chiral saturation of the same
  underlying flux.
  Inertia emerges as a secondary effect, reflecting the finite resolvability of
  changes in relational configurations rather than the presence of an intrinsic
  inertial property.

  The proposed framework does not modify Standard Model dynamics and does not
  introduce additional fundamental fields.
  Instead, it constrains the space of admissible effective descriptions by
  requiring that mass, charge, and inertia arise as saturated responses with
  well-defined high-gradient limits.
  Standard Model phenomenology is recovered as the unsaturated, linear-response
  regime of this description.

  We analyze the consistency of this interpretation with established
  phenomenology, its relation to bounded (Born--Infeld-type) effective dynamics,
  and the physical consequences implied by a finite projective capacity.
  The results suggest that mass and charge represent complementary limits of a
  single saturated relational mechanism, providing a unified and falsifiable
  structural basis for inertial, gravitational, and electromagnetic response.
}
