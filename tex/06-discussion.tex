\section{Discussion, Limitations, and Outlook}
  \label{sec:discussion}

  In this final section, we summarize the conceptual implications of the proposed
  framework, clarify its limitations, and outline possible directions for future
  investigation.
  While the analysis remains primarily structural, it now includes an explicit
  effective realization of the saturation mechanism.
  This allows physical constraints to be formulated at the level of an effective
  Lagrangian, rather than at the level of interpretation alone.

  \subsection{Conceptual implications}
    \label{subsec:implications}

    A central implication of the present analysis is that mass, electric charge, and
    inertia need not be treated as fundamentally distinct physical primitives.
    Instead, they may be understood as different symmetry realizations of a single
    structural mechanism, namely the saturation of effective response under
    non-injective projection.

    From this perspective, the apparent diversity of physical properties reflects
    differences in how the effective description responds to relational variation,
    rather than differences in underlying substance.
    Scalar, oriented, and finite-bandwidth response modes correspond respectively to
    mass, charge, and inertial behavior, with inertia arising as the linear-response
    limit of a bounded projective update process.

    The explicit effective construction introduced in
    Section~\ref{sec:formal_derivation} shows that these distinctions can be realized
    within a single bounded-response framework, without modifying the operational
    definitions of mass and charge.
    In particular, mass appears as a spectral gap induced by isotropic saturation,
    while electric charge emerges as the generator of a local phase degeneracy
    associated with oriented saturation.
    Inertia, in turn, is shown to correspond to the first-order expansion of a
    saturated update dynamics.

    This interpretation provides a unified conceptual basis for the equivalence
    between inertial and gravitational mass and for the signed nature of electric
    charge, while preserving their empirical roles within established physical
    theories.

  \subsection{Relation to existing foundational approaches}
    \label{subsec:relation}

    The framework developed here is compatible with a wide range of foundational
    approaches in which spacetime and physical observables are regarded as effective
    constructs.
    It does not rely on a specific microscopic ontology and can be embedded in
    different relational, background-independent, or pre-geometric settings.

    Unlike approaches that postulate new degrees of freedom or modified fundamental
    dynamics, the present work operates at the level of effective description.
    Its contribution is to isolate a minimal structural mechanism that constrains
    how familiar physical notions can consistently arise within effective theories.

    In this sense, the framework complements rather than replaces existing
    formulations.
    It provides an interpretative and constraining layer that may coexist with
    standard quantum field theory and relativistic dynamics, while clarifying the
    domain of validity of linear-response extrapolations.

  \subsection{Limitations of the present work}
    \label{subsec:limitations}

    The present analysis is subject to several important limitations.

    First, no microscopic dynamics of the underlying relational description is
    specified.
    As a result, the framework does not predict numerical values for particle masses,
    charges, or coupling constants.
    These remain empirical inputs determined by effective theories operating in the
    unsaturated regime.

    Second, the analysis does not provide quantitative predictions for high-energy or
    strong-field deviations from Standard Model behavior.
    However, the effective formalism developed in
    Section~\ref{sec:formal_derivation} identifies a well-defined functional departure
    from linear inertia and a characteristic acceleration scale associated with
    saturation, beyond which strictly linear extrapolations necessarily fail.

    Third, the framework does not address questions related to particle generations,
    flavor structure, or symmetry-breaking patterns within the Standard Model.
    Its scope is restricted to the structural origin and effective interpretation of
    mass, charge, and inertia.

    Finally, although the framework implies the existence of saturation regimes at
    extreme relational gradients, such regimes are only indirectly accessible in
    current experimental settings.
    Direct tests are therefore most naturally associated with future strong-field or
    high-gradient probes.

  \subsection{Possible extensions and outlook}
    \label{subsec:outlook}

    Despite these limitations, the framework suggests several concrete directions for
    further investigation.

    One natural extension is the construction of explicit relational or spectral
    models in which the projective bound and its associated saturation scale can be
    derived rather than postulated.
    Such developments could bridge the gap between structural necessity and
    quantitative prediction.

    Another important direction concerns the detailed study of strong-field and
    high-gradient regimes, where saturation effects are expected to become
    operational.
    Clarifying the relation between the present bounded-response framework and known
    nonlinear phenomena in quantum field theory may help identify experimentally
    relevant signatures.

    More broadly, the analysis invites reconsideration of the status of physical
    parameters traditionally regarded as fundamental.
    If mass and charge are effective descriptors tied to finite descriptive capacity,
    their role in physical theories may be understood as constrained rather than
    primitive.

  \subsection{Concluding remarks}
    \label{subsec:conclusion}

    We have proposed a unified structural interpretation of mass, electric charge, and
    inertia as effective manifestations of saturated response under non-injective
    projection.
    The framework is fully consistent with Standard Model phenomenology and does not
    modify its dynamical content in experimentally tested regimes.

    By providing an explicit effective realization of the saturation mechanism, the
    present work goes beyond purely interpretative unification and introduces
    well-defined physical constraints.
    In particular, the derivation of inertia as the linear limit of a bounded update
    process implies necessary high-gradient limits, rendering the framework
    falsifiable in principle.

    While deliberately modest in its quantitative claims, the analysis establishes a
    coherent theoretical basis for further exploration of the structural origin of
    physical properties.
