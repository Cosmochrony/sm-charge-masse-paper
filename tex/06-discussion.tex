\section{Discussion, Limitations, and Outlook}
  \label{sec:discussion}

  In this final section, we summarize the conceptual implications of the proposed
  framework, clarify its limitations, and outline possible directions for future
  investigation.
  While the analysis remains primarily structural, the emphasis is no longer on
  interpretation alone, but on the physical constraints implied by a bounded
  projective description.

  \subsection{Conceptual implications}
    \label{subsec:implications}

    A central implication of the present analysis is that mass, electric charge, and
    inertia need not be treated as fundamentally distinct physical primitives.
    Instead, they may be understood as different symmetry realizations of a single
    structural mechanism, namely the saturation of effective response under
    non-injective projection.

    From this perspective, the apparent diversity of physical properties reflects
    differences in how the effective description responds to relational variation,
    rather than differences in underlying substance.
    Scalar, oriented, and isotropic finite-response modes correspond respectively to
    mass, charge, and inertial behavior.

    This interpretation provides a unified conceptual basis for the equivalence
    between inertial and gravitational mass and for the signed nature of electric
    charge, without modifying their operational definitions.
    It also clarifies why these quantities enter physical laws as parameters
    governing response, rather than as directly observable entities.

  \subsection{Relation to existing foundational approaches}
    \label{subsec:relation}

    The framework developed here is compatible with a wide range of foundational
    approaches in which spacetime and physical observables are regarded as effective
    constructs.
    It does not rely on a specific microscopic ontology and can be embedded in
    different relational, background-independent, or pre-geometric settings.

    Unlike approaches that postulate new degrees of freedom or modified dynamics,
    the present work operates at the level of description.
    Its contribution is to isolate a minimal structural mechanism that constrains
    how familiar physical notions can consistently arise within effective theories.

    In this sense, the framework complements rather than replaces existing theories.
    It provides an interpretative and constraining layer that may coexist with
    standard formulations of quantum field theory and relativity, while clarifying
    the domain of validity of linear-response extrapolations.

  \subsection{Limitations of the present work}
    \label{subsec:limitations}

    The present analysis is subject to several important limitations.

    First, no microscopic dynamics of the underlying relational description is
    specified.
    As a result, the framework does not predict numerical values for particle masses,
    charges, or coupling constants.
    These remain empirical inputs determined by effective theories.

    Second, the analysis is primarily structural.
    While logical necessity and consistency arguments identify regimes in which
    linear extrapolations must fail, no explicit calculation of high-energy
    deviations is provided here.

    Third, the framework does not address questions related to particle generations,
    flavor structure, or symmetry-breaking patterns within the Standard Model.
    Its scope is restricted to the conceptual and structural status of mass, charge,
    and inertia as effective descriptors.

    Finally, although the framework implies the existence of saturation regimes at
    extreme gradients, such regimes remain experimentally inaccessible in current
    particle physics experiments.
    Direct tests are therefore deferred to future high-field or high-energy
    contexts.

  \subsection{Possible extensions and outlook}
    \label{subsec:outlook}

    Despite these limitations, the framework suggests several concrete directions
    for further investigation.

    One natural extension is the development of explicit models in which the
    projective bound and its associated saturation can be derived within concrete
    relational or spectral constructions.
    Such models could bridge the gap between structural necessity and quantitative
    predictions.

    Another important direction concerns the detailed study of strong-field and
    high-gradient regimes, where saturation effects are expected to become
    operational.
    Clarifying the relation between the present framework and known bounded-response
    phenomena in quantum field theory may help identify experimentally relevant
    signatures.

    More broadly, the analysis invites reconsideration of the status of physical
    parameters traditionally regarded as fundamental.
    If mass and charge are effective descriptors tied to finite descriptive capacity,
    their role in physical theories may be understood as constrained rather than
    primitive.

  \subsection{Concluding remarks}
    \label{subsec:conclusion}

    We have proposed a unified structural interpretation of mass, electric charge,
    and inertia as effective manifestations of saturated response under non-injective
    projection.
    The framework is fully consistent with Standard Model phenomenology and does not
    modify its dynamical content in tested regimes.

    By identifying necessary high-gradient limits implied by bounded projective
    descriptions, the analysis goes beyond interpretative unification and introduces
    clear physical constraints.
    While deliberately modest in its quantitative claims, the framework provides a
    coherent and falsifiable basis for further exploration of the structural origin
    of physical properties.
