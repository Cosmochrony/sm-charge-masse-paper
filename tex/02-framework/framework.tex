\section{Relational Framework and Projective Description}
  \label{sec:framework}

  In this section, we summarize the minimal relational and projective framework required for the
  present analysis.
  The detailed construction of this framework is developed elsewhere.
  Here we restrict attention to the structural assumptions relevant for the interpretation of mass,
  charge, and inertia.

  \subsection{Underlying relational description}
    \label{subsec:relational-substrate}

    We assume that effective spacetime observables arise from an underlying relational description,
    in which configurations encode admissible correlations between abstract degrees of freedom.
    No spacetime manifold, metric, or field content is assumed at this level.

    Explicit realizations of such relational descriptions, including spectral reconstructions of
    effective geometry from correlation structure alone, have been developed in companion work~\cite{Beau2026a}.
    The present paper does not rely on the details of those constructions, but adopts their structural
    conclusion: spacetime geometry is an effective description, not a fundamental input.

    The underlying relational description should not be interpreted as a hidden-variable completion
    of existing theories.
    It defines a space of admissible configurations whose effective description is intrinsically
    coarse-grained.

    Configurations at this level are not endowed with local coordinates, temporal ordering, or intrinsic
    geometric meaning.
    These notions arise only when relational configurations admit a stable and approximately injective
    representation.

  \subsection{Projection to effective observables}
    \label{subsec:projection}

    Observable physical quantities are defined through a projection
    \begin{equation}
      \Pi : \Omega \rightarrow \mathcal{O},
    \end{equation}
    where $\Omega$ denotes the space of underlying relational configurations and $\mathcal{O}$ the space
    of effective observables accessible within spacetime descriptions.

    A key structural feature of this projection is that it is generally non-injective.
    Distinct underlying configurations may correspond to the same effective observable state.
    This identification is not an approximation, but an intrinsic property of the descriptive mapping.

    The physical consequences of non-injective projection have been analyzed in detail in the context
    of quantum correlations~\cite{Beau2026b}.
    There it was shown that non-injectivity provides a sufficient mechanism for the failure of classical
    probabilistic factorization, without invoking nonlocal dynamics or hidden-variable assumptions.

    In the present work, the same projective structure is applied to inertial and interaction-related
    quantities.
    Mass, charge, and inertia are thus treated as effective descriptors defined on equivalence classes
    of underlying relational configurations.

  \subsection{Saturation and bounded response}
    \label{subsec:saturation}

    A further assumption, motivated by both structural and dynamical considerations, is that the
    projection $\Pi$ admits an intrinsic saturation.
    Beyond a certain threshold, additional variations in the underlying relational configuration
    cannot be resolved within the effective description.

    This bounded-response behavior has been studied in detail in companion work~\cite{Beau2026c},
    where it was shown that saturation naturally leads to effective Born--Infeld-type dynamics and to
    the dynamical selection of stable geometric regimes.

    In the present paper, saturation is not introduced as a modification of fundamental dynamics.
    It is interpreted as a limitation of the effective description itself, reflecting finite
    projective resolution.

    Operationally, this implies that effective responses cease to scale linearly in regimes where the
    underlying relational gradients exceed the resolving capacity of the projection.

  \subsection{Effective descriptors and regime dependence}
    \label{subsec:effective-descriptors}

    Physical quantities such as mass, electric charge, and inertia are treated as effective descriptors
    defined within the projected space $\mathcal{O}$.
    They characterize how the effective description responds to variations of the underlying relational
    configuration.

    These quantities are inherently regime dependent.
    In domains where the projection is approximately injective and unsaturated, classical descriptions
    are recovered and effective responses behave linearly.
    In saturated regimes, departures from linearity arise as a direct consequence of projective
    limitations.

    Importantly, the present framework does not modify the operational content of the Standard Model.
    It provides a structural reinterpretation of existing quantities, compatible with established
    phenomenology.

    Within this perspective, distinctions between mass, charge, and inertial response correspond to
    different symmetry classes of saturated effective behavior.
    This classification is developed in the following sections.

  \subsection{Scope and relation to companion work}
    \label{subsec:scope}

    This paper is self-contained at the conceptual level.
    It does not require familiarity with the detailed constructions presented in the companion papers,
    but relies on their results as structural input.

    Paper~A~\cite{Beau2026a} establishes the relational reconstruction of effective spacetime geometry.
    Paper~B~\cite{Beau2026b} analyzes the consequences of non-injective projection for probabilistic
    descriptions.
    Paper~C~\cite{Beau2026c} develops the theory of bounded relaxation and saturation.

    The present work builds on these results to address the origin of mass, charge, and inertia as
    effective descriptors.
    In the next section, we analyze symmetry classes of saturated responses and their physical
    interpretation.
