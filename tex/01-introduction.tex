\section{Introduction}
  \label{sec:introduction}

  The physical notions of mass, electric charge, and inertia occupy a central role
  in modern physics.
  They enter as fundamental parameters in both classical and quantum theories,
  yet their conceptual origin remains largely unexplained.

  In the Standard Model, mass and charge are introduced as intrinsic properties of
  elementary fields, with numerical values fixed by empirical input.
  While the Higgs mechanism provides a dynamical account of mass generation for
  gauge bosons and fermions, it does not address why mass exists as a property, nor
  why inertial response accompanies it~\cite{Higgs1964}.
  Electric charge, similarly, is treated as a primitive coupling constant,
  constrained by gauge symmetry but not derived from a deeper structural
  principle~\cite{PeskinSchroeder}.

  From a foundational perspective, inertia presents an additional conceptual
  challenge.
  The resistance of a system to acceleration is usually taken as a defining
  feature of matter, yet its relation to gravitational mass and its possible
  emergence from more primitive structures have been the subject of long-standing
  debate~\cite{Mach1893,Einstein1916}.

  A parallel line of inquiry has emerged in approaches to quantum gravity and
  pre-geometric physics.
  In these frameworks, spacetime geometry, and sometimes even locality and
  causality, are not assumed as fundamental, but are reconstructed as effective
  descriptions from more primitive relational or algebraic
  structures~\cite{Rovelli2004,AmelinoCamelia2013}.
  Within such approaches, familiar physical quantities may acquire an emergent or
  regime-dependent status.

  Recent work has emphasized that effective physical descriptions often arise
  through projections from an underlying configuration space to observable
  spacetime variables.
  When such projections are non-injective, multiple underlying configurations
  correspond to the same effective observable state, leading to intrinsic
  information loss at the descriptive level~\cite{Beau2026b}.
  This structural feature has been shown to account for the breakdown of classical
  probabilistic factorization in quantum correlations, without invoking nonlocal
  dynamics~\cite{Bell1964}.

  In parallel, bounded-response mechanisms have been extensively studied in
  effective field theories.
  Born--Infeld electrodynamics provides a paradigmatic example in which divergences
  are regulated by intrinsic saturation of the field
  response~\cite{BornInfeld1934}.
  Related saturation phenomena appear in gravitational and cosmological contexts,
  where effective responses cease to scale linearly beyond certain thresholds.

  Motivated by these developments, we examine whether mass, electric charge, and
  inertial response can be understood as manifestations of a single structural
  mechanism, and whether such an interpretation imposes non-trivial constraints on
  effective physical descriptions.
  Specifically, we consider a pre-geometric relational framework in which physical
  observables arise through a projection subject to intrinsic saturation bounds.

  Within this setting, mass is interpreted as an isotropic inhibition of
  relational relaxation, while electric charge corresponds to an oriented or
  asymmetric saturation of the same underlying relational flux.
  Inertia then emerges as the linear-response limit of a bounded projective
  update process, reflecting the finite resolvability of changes in relational
  configurations rather than a primitive dynamical property.

  The purpose of this work is not to modify the Standard Model or to introduce new
  fundamental fields.
  Instead, it is to identify structural constraints on how familiar physical
  quantities can consistently arise as effective descriptors within a
  projection-limited framework.
  In addition to the conceptual analysis, we provide an explicit effective
  realization of the proposed saturation mechanism in Lagrangian form, showing
  that bounded relational projection naturally induces Born--Infeld--type
  nonlinearities and associated symmetry structures.

  The paper is organized as follows.
  Section~\ref{sec:framework} introduces the relational and projective framework
  underlying the analysis.
  Section~\ref{sec:saturation} develops the classification of saturated effective
  responses and shows how mass, electric charge, and inertia arise as distinct
  symmetry realizations within this framework.
  Section~\ref{sec:formal_derivation} provides a formal effective derivation of the
  saturation mechanism and its physical consequences.
  Section~\ref{sec:consistency} examines the consistency of this interpretation with
  Standard Model phenomenology and identifies necessary high-gradient limits.
  We conclude in Section~\ref{sec:discussion} with a discussion of conceptual
  implications, limitations, and possible directions for further investigation.
